\newenvironment{caputFesti}{\begin{center}}{\end{center}}

% pieces of the title of a feast
\newcommand{\diesFesti}[1]{{\small #1} \vspace{2mm}\\}
\newcommand{\nomenFesti}[1]{\textbf{\Large #1}\vspace{3mm}\\}
\newcommand{\descriptioFesti}[1]{\textbf{#1}\\}
\newcommand{\dignitasFesti}[1]{\small #1}

% headings of various parts
\newcommand{\hora}[1]{\vspace{5mm}\noindent\textbf{#1}\vspace{2mm}}
\newcommand{\horaVesperaeI}{\hora{In I Vesperis}}
\newcommand{\horaLaudes}{\hora{In Laudibus}}
\newcommand{\horaLaudesEtHorae}{\hora{Ad Laudes et per Horas}}
\newcommand{\horaTertia}{\hora{Ad Tertiam}}
\newcommand{\horaSexta}{\hora{Ad Sextam}}
\newcommand{\horaNona}{\hora{Ad Nonam}}
\newcommand{\horaVesperaeII}{\hora{In II Vesperis}}

\newcommand{\parsHorae}[1]{\textbf{#1}\vspace{1mm}}
\newcommand{\parsOratio}{\begin{center}\emph{Oratio}\end{center}}
\newcommand{\parsCapitulum}[1]{\hspace{1.2cm}\textsc{Capitulum.}\hfill #1\hspace{1cm}}

\newcommand{\rubrica}[1]{{\small \emph{#1}}}

\newcommand{\utInBreviarioPraeter}{%
  \rubrica{Omnia ut in Antiphonario sub hac die, praeter sequentia.}}
\newcommand{\perDominum}{Per Dóminum.}

\newcommand{\versiculus}[2]{\noindent ℣. #1\\℟. #2}

% calendar
\newcommand{\calMonth}[1]{\begin{center}\textbf{#1}\end{center}}

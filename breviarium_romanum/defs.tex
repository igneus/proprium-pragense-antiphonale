\newcommand{\hruleIfNotPageBeginning}{%
  % https://tex.stackexchange.com/a/114862/13146
  \par
  \cleaders\vbox to 2\baselineskip{%
    \vss
    \hrule width\linewidth
    \vss
  }\vskip2\baselineskip
}

\newenvironment{caputFesti}{%
  \hruleIfNotPageBeginning
  \begin{center}}{\end{center}}

% pieces of the title of a feast
\newcommand{\diesFesti}[1]{{\small #1} \vspace{2mm}\\}
\newcommand{\nomenFesti}[1]{\textbf{\Large #1}\vspace{3mm}\\}
\newcommand{\descriptioFesti}[1]{\textbf{#1}\\}

\newcommand{\dignitasFesti}[1]{\small #1}
\newcommand{\dignitasI}{\dignitasFesti{I Classis}}
\newcommand{\dignitasII}{\dignitasFesti{II Classis}}
\newcommand{\dignitasIII}{\dignitasFesti{III Classis}}

\newcommand{\dioecesis}[1]{\emph{#1}\\}
\newcommand{\dioecPraga}{\dioecesis{In archidioecesi Pragensi}}
\newcommand{\dioecBud}{\dioecesis{In dioecesi Budvicensi}}

% headings of various parts
\newcommand{\hora}[1]{\vspace{5mm}\noindent\textbf{#1}\vspace{2mm}}
\newcommand{\horaVesperaeI}{\hora{In I Vesperis}}
\newcommand{\horaNocturnusI}{\hora{In I Nocturno}}
\newcommand{\horaNocturnusII}{\hora{In II Nocturno}}
\newcommand{\horaNocturnusIII}{\hora{In III Nocturno}}
\newcommand{\horaLaudes}{\hora{In Laudibus}}
\newcommand{\horaLaudesEtHorae}{\hora{Ad Laudes et per Horas}}
\newcommand{\horaTertia}{\hora{Ad Tertiam}}
\newcommand{\horaSexta}{\hora{Ad Sextam}}
\newcommand{\horaNona}{\hora{Ad Nonam}}
\newcommand{\horaVesperaeII}{\hora{In II Vesperis}}

\newcommand{\parsHorae}[1]{\textbf{#1}\vspace{1mm}}
\newcommand{\parsOratio}{\begin{center}\emph{Oratio}\end{center}}
\newcommand{\parsCapitulum}[1]{\hspace{1.2cm}\textsc{Capitulum.}\hfill #1\hspace{1cm}}
\newcommand{\parsLectio}[1]{\begin{center}\emph{Lectio #1}\end{center}}
\newcommand{\parsLectioAnnot}[2]{\begin{center}\emph{Lectio #1}\\\footnotesize{#2}\end{center}}
\newcommand{\parsLectioEv}[1]{\noindent\emph{Lectio vii \hfill #1}}
\newcommand{\parsEvangelium}[1]{%
  \vspace{3mm}Léctio sancti Evangélii secúndum #1.}
\newcommand{\inIlloTempore}{\lettrine{I}{n} illo témpore:}
\newcommand{\parsHomilia}[2]{%
  \begin{center}Homilía #1\\{\footnotesize\emph{#2}}\end{center}}
\newcommand{\parsSermo}[2]{%
  \begin{center}Sermo #1\\{\footnotesize\emph{#2}}\end{center}}
\newcommand{\parsTeDeum}{\textbf{T}e Deum.}

\newcommand{\rubrica}[1]{{\small \emph{#1}}}
\newcommand{\rubRespDeCommuni}[1]{%
  \rubrica{Responsoria} #1 \rubrica{etc. de Communi.}}

\newcommand{\utInBreviarioPraeter}{%
  \rubrica{Omnia ut in Antiphonario sub hac die, praeter sequentia.}}
\newcommand{\perDominum}{Per Dóminum.}

\newcommand{\vv}{℣.}
\newcommand{\rr}{℟.}
\newcommand{\versiculus}[2]{\noindent ℣. #1\\℟. #2}

% calendar
\newcommand{\calMonth}[1]{\begin{center}\textbf{#1}\end{center}}

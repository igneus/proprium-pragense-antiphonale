\begin{caputFesti}
  \diesFesti{Die 2 Maji}
  \nomenFesti{S. Sigismundi}
  \descriptioFesti{Regis, Martyris et Patroni Secundarii Regni}
  \dignitasFesti{Duplex majus}
\end{caputFesti}

% in the old Officia propria antiphon and verse from the Common
% is printed.

\parsOratio
Deus, qui hunc diem beáti Sigismúndi passióne consecrásti:
praesta, quaésumus;
ut, cujus sollémnia celebrámus in terris,
ejus apud te suffrágiis adjuvémur in caelis.
(Per Dóminum.)

\rubrica{Deinde Commemoratio S. Athanasii Episcopi, Confessoris
  et Ecclesiae Doctoris
  per Antiphonam}
O Doctor óptime.
\rubrica{et Versum}
Amávit eum.

\parsOratio
Exáudi, quaésumus, Dómine, preces nostras,
quas in beáti Athanásii, Confessóris tui atque Pontíficis,
sollemnitáte deférimus:
et, qui tibi digne méruit famulári,
ejus intercedéntibus méritis,
ab ómnibus nos absólve peccátis.
Per Dóminum.

\horaNocturnusII

\parsLectio{iv}

\lettrine{S}{igismúndus,} Gunebáldi Burgundiórum Regis fílius,
a téneris annis in cathólica religióne educátus,
ita Christiánae fídei fuit addíctus,
ut adoléscens factus, diu noctúque vigíliis, jejúniis et oratiónibus vacans,
non obscúra déderit sanctitátis indícia.
Mórtuo Gunebáldo, patérnum sceptrum adéptus,
zelo propagándae fídei succénsus,
summa cura summóque labóre, praesértim vero exémplo et virtútibus
regnum a ténebris infidelitátis ad lucem veritátis addúcere stúduit.
Religiónis quoque, ac devotiónis erga divínum cultum
et Sanctórum Ecclésias promovéndae causa,
monastérium Agaunénse, in quo Psalléntium Ordinem instítuit,
regáli munificéntia cum dómibus Basilicísque aedificávit,
et magnis redítibus locupletávit.

\rubRespDeCommuni{Lux perpétua.}

\parsLectio{v}

\lettrine{A}{míssa} prióri cónjuge fília Theodoríci Regis Itáliae,
ex qua fílium suscéperat nómine Sigerícum,
áliam duxit uxórem.
Hujus nequíssimis decéptus suasiónibus, fílium intérfici jussit.
Quo facto, Sigismúndus corde compúnctus, super cadáver próruens,
flere coepit amaríssime: deínde velut alter David,
severiórem poeniténtiae viam ingréssus, multis Sanctórum locis perlustrátis,
demum labóribus et inédia fessus, ad sepúlchra Sanctórum Agaunénsium pervéniens,
ibi per multos dies in fletu et jejúnio persevérans,
Deum eníxe rogábat, ut si quid adhuc pro consequéndo coeléstis pátriae
regno sibi superésset, misericórditer osténdere dignarétur.
Summa Dei bónitas, quae labóribus servi sui remuneratiónem
diútius non patiebátur différre,
ad palmam martýrii ipsum eátenus vocávit,
ut sanctórum Thebaeórum Mártyrum collégio,
quorum se offício in Dei láudibus sociáverat devotióne,
paradísi quoque sociarétur glória.

\parsLectio{vi}

\lettrine{N}{am} cum Franci Galliárum gentes et urbes depopularéntur,
Burgundiónibus, qui adhuc in infidelitáte persistébant,
sibi sociátis, Sigismúndum, qui ut barbarórum ferocitátem eváderet,
Vesállis montem petíerat, ibíque tonso crine,
et Religiónis hábitu suscépto, singuláriter habitábat,
a suis decéptum, et ad sepúlchra Sanctórum ductum,
una cum uxóre et fíliis capitáli senténtia adjudicátum,
in púteum véterem apud Colóniam vicum projecérunt.
Corpus ejus divína póstea revelatióne patefáctum,
indéque sublátum, et in Ecclésia Agaunénsi honorífice sepúltum,
miráculis claréscere coepit, ita Deo ejus sanctitátem comprobánte.
Póstea a Carólo Quarto Cáesare
ad Ecclésiam Metropolitánam Pragénsem translátum,
gloriósum ibídem túmulum invénit.

\horaNocturnusIII

\parsEvangelium{Joánnem}

\parsLectioEv{Cap. 15, 1-7}

\inIlloTempore{}
Dixit Jesus discípulis suis:
Ego sum vitis vera, et Pater meus agrícola est.
Et réliqua.

\parsHomilia{sancti Augustíni Epíscopi}{Tract. 81 in Joannem}

\lettrine{V}{item} se dixit esse Jesus,
et discípulos suos pálmites,
et agrícolam Patrem:
unde jam pridem, sicut potúimus, disputátum est.
In hac autem lectióne cum adhuc de se ipso, qui est vitis,
et de suis palmítibus, hoc est, discípulis, loquerétur:
Manéte, inquit, in me, et ego in vobis.
Non eo modo illi in ipso, sicut ipse in illis.
Utrúmque autem prodest non ipsi, sed illis:
ita quippe in vite sunt pálmites,
ut viti non cónferant, sed inde accípiant, unde vivant;
ita vero vitis in palmítibus,
ut vitále aliméntum subminístret eis, non sumat ab eis.

\rubRespDeCommuni{Ego sum vitis vera.}

\rubrica{Feria III et VI infra hebdomadam I et II post Octavam Paschae,
  quoties in I Nocturno Lectiones fuerint de Scriptura occurrenti
  cum suis Responsoriis de Tempore,
  loco VII Responsorii dicitur}
\rr{} Tristítia vestra, allelúja.
\rubrica{ut in III Nocturno Dominicae III post Pascham.}

\parsLectio{viii}

\lettrine{A}{c} per hoc, et manéntem in se habére Christum,
et manére in Christo, discípulis prodest utrúmque, non Christo.
Nam praecíso pálmite, potest de viva radíce álius polluláre;
qui autem praecísus est, sine radíce non potest vívere.
Dénique adjúngit, et dicit:
Sicut palmes non potest ferre fructum a semetípso,
nisi mánserit in vite;
sic nec vos, nisi in me manséritis.
Magna grátiae commendátio, fratres mei:
corda ínstruit humílium, ora óbstruit superbórum.

\rubrica{Pro S. Athanasio Episcopo, Confessore et Ecclesiae Doctore:}

\parsLectio{ix}

\lettrine{A}{thanásius,} epíscopus Alexandrínus,
cathólicae religiónis propugnátor acérrimus,
cum, adhuc diáconus, in Concílio Nicaéno Aríi impietátem repressísset,
tantum ódium Arianórum suscépit,
ut ex eo témpore et insídias molíri numquam destíterint.
In exsílium actus, in Gállia apud Tréviros exulávit.
Incredíbiles dein calamitátes perpéssus,
magnam orbis partem peragrávit;
ac saepe e sua ecclésia ejéctus, saepe étiam in eándem,
Júlii Románi Pontíficis auctoritáte atque decrétis concílii
Sardicénsis ac Jerosolymitáni restitútus est,
Ariánis intérea illi semper inféstis.
Dénique ex tot tantísque perículis divínitus eréptus,
Alexandríae mórtuus est sub Valénte.
Ejus vita et mors magnis nobilitáta est miráculis.
Multa pie et ad illustrándam cathólicam fidem praecláre scripsit,
sexque et quadragínta annos in summa témporum varietáte
Alexandrínam ecclésiam sanctíssime gubernávit.

\parsTeDeum

\rubrica{In Laudibus fit Commemoratio S. Athanasii
  per Antiphonam}
Euge, serve bone.
\rubrica{et Versum}
Justum dedúxit.
\rubrica{Oratio ut supra.}

\rubrica{Vesperae de sequenti, Commemoratio S. Sigismundi Martyris
  per Antiphonam}
Sancti et justi.
\rubrica{et Versum}
Pretiósa in conspéctu.
\rubrica{Oratio ut supra.}

\rubrica{Deinde fit Commemoratio S. Athanasii ut in Breviario.}

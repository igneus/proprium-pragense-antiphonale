\begin{caputFesti}
  \diesFesti{Die 6 Decembris}
  \dioecBud
  \nomenFesti{S. Nicolai}
  \descriptioFesti{Episcopi et Confessoris, Titularis Ecclesiae Cathedralis}
  \dignitasFesti{Duplex 1 classis cum Octava communi}
\end{caputFesti}

\rubrica{Omnia etiam Lectiones 1 Nocturni}
Fidélis sermo,
\rubrica{de Communi Confessoris Pontificis,
  praeter ea, quae in Breviario habentur propria.}

\rubrica{In Laudibus Commemoratio Feriae; in II Vesperis
  Commemoratio sequentis et Feriae.}

\begin{caputFesti}
  \diesFesti{Die 13 Decembris}
  \dioecBud
  \nomenFesti{In Octava S. Nicolai}
  \descriptioFesti{Episcopi et Confessoris}
  \dignitasFesti{Duplex majus}
\end{caputFesti}

\rubrica{In I Vesperis fit Commemoratio
  Octavae Immaculatae Conceptionis
  et S. Luciae Virginis et Martyris et Feriae.}

\horaNocturnusII

\parsSermo{sancti Gregórii Papae}{Part. 2 Pastoralis, cap. 1}

\parsLectio{iv}

\lettrine{T}{antum} debet actiónem pópuli áctio transcéndere praésulis,
quantum distáre solet a grege vita pastóris.
Opórtet namque, ut metíri se sollícite stúdeat,
quanta tenéndae rectitúdinis necessitáte constríngitur,
sub cujus aestimatióne pópulus grex vocátur.
Sit ergo necésse est cogitatióne mundus,
actióne praecípuus,
discrétus in siléntio,
útilis in verbo,
síngulis compassióne próximus,
prae cunctis contemplatióne suspénsus,
bene agéntibus per humilitátem sócius,
contra delinquéntium vítia per zelum justítiae eréctus,
internórum curam in exteriórum occupatióne non mínuens,
exteriórum providéntiam in internórum sollicitúdine non relínquens.

\rubRespDeCommuni{Invéni David.}

\parsLectioAnnot{v}{Part. 1 Pastor. Cap. 9 et 10}

\lettrine{C}{onsiderándum} quoque est, quia cum curam pópuli eléctus
praésul súscipit, quasi ad aegrum médicus accédit.
Si ergo adhuc in ejus córpore passiónes vivunt,
qua praesumptióne percússum medéri próperat,
qui in fácie vulnus portat?
Ille ígitur modis ómnibus debet ad exémplum bene vivéndi pértrahi,
qui cunctis carnis passiónibus móriens,
jam spiritáliter vivit,
qui próspera mundi postpónit,
qui nulla advérsa pertiméscit,
qui sola intérna desíderat;
cujus intentióni bene cóngruens,
nec omníno per imbecillitátem corpus,
nec valde per contumáciam repúgnat spíritus:
qui ad aliéna cupiénda non dúcitur,
sed própria largítur.

\parsLectioAnnot{vi}{Ibidem Capite 8}

\lettrine{U}{nde} ipsum quoque Episcopátus offícium boni óperis
expressióne definítur, cum dícitur:
Si quis Episcopátum desíderat, bonum opus desíderat.
Ipse ergo sibi testis est, quia Episcopátum non áppetit,
qui non per hunc boni óperis ministérium,
sed honóris glóriam quaerit.
Sacrum quippe offícium non solum non díligit omníno,
sed nescit, qui ad culmen regíminis anhélans,
in occúlta meditatióne cogitatiónis,
ceterórum subjectióne páscitur,
laude própria laetátur,
ad honórem cor élevat,
rerum affluéntium abundántia exsúltat.
Mundi ergo lucrum quaéritur sub ejus honóris spécie,
quo mundi déstrui lucra debúerant.

\horaNocturnusIII

\parsEvangelium{Matthaéum}

\parsLectioEv{Cap. 25, 14-23}

\inIlloTempore{}
Dixit Jesus discípulis suis parábolam hanc:
Homo péregre proficíscens, vocávit servos suos,
et trádidit illis bona sua.
Et réliqua.

\parsHomilia{sancti Joánnis Chrysostómi}{Ex Hom. 79 in Matth. circa med.}

Talénta hic pro eo, quod unusquísque fácere potest,
accípimus, sive auctoritáte protégere,
sive pecúniis juváre,
sive doctrína admonére,
sive ália quápiam re proxímis prodésse queas.
Nemo secum dicat: Cum unum taléntum hábeam,
nihil possum effícere: potes profécto ex ea una re sola comprobári.
Non es paupérior illa vídua, non es Petro atque Joánne rustícior:
qui quamvis rudes simul atque illiteráti fuérunt,
quóniam magno stúdio commúnem utilitátem compléxi fúerant,
caelórum príncipes facti sunt.

\rubRespDeCommuni{Amávit eum.}

\parsLectio{viii}

\lettrine{Q}{uippe} nulla res Deo grátior est,
quam ut univérsam vitam ad commúne cómmodum cónferas.
Idcírco ratióne atque oratióne nos Deus decorávit,
mentem et ingénium concéssit,
manus, pedes, vires córporis dedit,
ut his ómnibus et nos ipsos et próximos tutémur.
Non enim ad agéndas grátias solúmmodo confert orátio,
verum étiam ad docéndum, ad admonéndum perútilis est:
qua in re si ea diligénter utémur Dóminum;
sin vero in contráriis, diábolum immitábimur.

\rubrica{Pro S. Lucia Virgine et Martyre:}

\parsLectio{ix}

\lettrine{L}{úcia}, virgo Syracusána,
génere et christiána fide nóbilis,
Cátanam ad beátae Agathae sepúlcrum,
Eutýchiae matris, sánguinis fluxu laborántis,
sanitátem impetrávit.
Mox bona ómnia quae in dotem esset acceptúra,
a matre impetráta, paupéribus distríbuit.
Quare apud Paschásium preféctum, quod christiána esset, accusáta,
nec blandítiis nec minis addúci pótuit, ut idólis sacrificáret.
Tunc Paschásius ira inflammátus, Lúciam eo trahi jussit,
ubi ejus virgínitas violarétur;
sed divínitus factum est, ut firma Virgo ita consísteret,
ut nulla vi de loco dimovéri posset.
Quam ob rem praeféctus circum ipsam ignem accéndi imperávit:
sed cum ne flamma quidem eam laéderet,
multis torméntis excruciátae guttur gládio transfígitur.
Quo vúlnere accépto, Lúcia, praedícens Ecclésiae tranquillitátem,
quae futúra erat Diocletiáno et Maximiáno mórtuis,
Idibus Decémbris spíritum Deo réddidit.
Cujus corpus Syracúsis sepúltum,
deínde Constantinópolim, postrémo Venétias translátum est.

\parsTeDeum

\rubrica{In Laudibus fit Commemoratio S. Luciae,
  Octavae Immaculatae Conceptionis
  et Feriae.}

\rubrica{In II Vesperis fit
  Commemoratio Immaculatae Conceptionis e I Vesperis Festi,
  S. Luciae
  et Feriae.}

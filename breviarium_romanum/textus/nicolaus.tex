\begin{caputFesti}
  \diesFesti{Die 6 Decembris}
  \dioecBud
  \nomenFesti{S. Nicolai}
  \descriptioFesti{Episcopi et Confessoris, Titularis Ecclesiae Cathedralis}
  \dignitasFesti{Duplex 1 classis cum Octava communi}
\end{caputFesti}

\rubrica{Omnia etiam Lectiones 1 Nocturni}
Fidélis sermo,
\rubrica{de Communi Confessoris Pontificis,
  praeter ea, quae in Breviario habentur propria.}

\rubrica{In Laudibus Commemoratio Feriae; in II Vesperis
  Commemoratio sequentis et Feriae.}

\begin{caputFesti}
  \diesFesti{Die 13 Decembris}
  \dioecBud
  \nomenFesti{In Octava S. Nicolai}
  \descriptioFesti{Episcopi et Confessoris}
  \dignitasFesti{Duplex majus}
\end{caputFesti}

\rubrica{In I Vesperis fit Commemoratio
  Octavae Immaculatae Conceptionis
  et S. Luciae Virginis et Martyris et Feriae.}

\horaNocturnusII

\parsSermo{sancti Gregórii Papae}{Part. 2 Pastoralis, cap. 1}

\parsLectio{iv}

\lettrine{T}{antum} debet actiónem pópuli áctio transcéndere praésulis,
quantum distáre solet a grege vita pastóris.
Opórtet namque, ut metíri se sollícite stúdeat,
quanta tenéndae rectitúdinis necessitáte constríngitur,
sub cujus aestimatióne pópulus grex vocátur.
Sit ergo necésse est cogitatióne mundus,
actióne praecípuus,
discrétus in siléntio,
útilis in verbo,
síngulis compassióne próximus,
prae cunctis contemplatióne suspénsus,
bene agéntibus per humilitátem sócius,
contra delinquéntium vítia per zelum justítiae eréctus,
internórum curam in exteriórum occupatióne non mínuens,
exteriórum providéntiam in internórum sollicitúdine non relínquens.

\rubRespDeCommuni{Invéni David.}

\parsLectioAnnot{v}{Part. 1 Pastor. Cap. 9 et 10}

\lettrine{C}{onsiderándum} quoque est, quia cum curam pópuli eléctus
praésul súscipit, quasi ad aegrum médicus accédit.
Si ergo adhuc in ejus córpore passiónes vivunt,
qua praesumptióne percússum medéri próperat,
qui in fácie vulnus portat?
Ille ígitur modis ómnibus debet ad exémplum bene vivéndi pértrahi,
qui cunctis carnis passiónibus móriens,
jam spiritáliter vivit,
qui próspera mundi postpónit,
qui nulla advérsa pertiméscit,
qui sola intérna desíderat;
cujus intentióni bene cóngruens,
nec omníno per imbecillitátem corpus,
nec valde per contumáciam repúgnat spíritus:
qui ad aliéna cupiénda non dúcitur,
sed própria largítur.

\parsLectioAnnot{vi}{Ibidem Capite 8}

\lettrine{U}{nde} ipsum quoque Episcopátus offícium boni óperis
expressióne definítur, cum dícitur:
Si quis Episcopátum desíderat, bonum opus desíderat.
Ipse ergo sibi testis est, quia Episcopátum non áppetit,
qui non per hunc boni óperis ministérium,
sed honóris glóriam quaerit.
Sacrum quippe offícium non solum non díligit omníno,
sed nescit, qui ad culmen regíminis anhélans,
in occúlta meditatióne cogitatiónis,
ceterórum subjectióne páscitur,
laude própria laetátur,
ad honórem cor élevat,
rerum affluéntium abundántia exsúltat.
Mundi ergo lucrum quaéritur sub ejus honóris spécie,
quo mundi déstrui lucra debúerant.

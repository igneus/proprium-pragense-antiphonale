\begin{caputFesti}
  \diesFesti{Die 1. Decembris}
  \dioecPraga
  \nomenFesti{B. Edmundi Campion}
  \descriptioFesti{Martyris e S. J.}
  \dignitasFesti{Semiduplex}
\end{caputFesti}

\rubrica{In II Vesperis praecedentis fit Com.
  per antiphonam}
Iste sanctus pro lege.
\rubrica{et versum}
Glória et honóre.

\parsOratio
Deus, qui verae fídei
et Sedis Apostólicae primátui propugnándo
beátum Mártyrem tuum Edmúndum
invícta fortitúdine roborásti:
ejus précibus exorátus,
nostrae, quaésumus, infirmitáti succúrre;
ut fortes in fide adversário resístere usque in finem valeámus.
Per Dóminum.

\rubrica{Et fit commemoratio feriae.}

\horaNocturnusII

\parsLectio{iv}

\lettrine{E}{dmúndus} Cámpion Londínii in Anglia natus,
humanióribus lítteris in Oxoniénsi academía,
tum sacris disciplínis Duáci in Anglórum seminário óptime excúltus,
Romae demum Societáti Jesu nomen dedit,
Pragae inter novítios recéptus,
primus Congregatiónis Beátae Maríae Vírginis praeses eléctus est,
et sacerdótio auctus per duos annos sacros sermónes habébat.
In pátriam Summi Pontíficis jussu,
una cum Robérto Persons, revérsus,
paucis ménsibus ea perfécit exémplo vitae,
excellénti doctrína atque agéndi dexteritáte,
ut ómnium ad se ánimos convérterit,
catholicórum quidem, ut eum audírent servaréntque,
inimicórum ut pérderent.

\rubRespDeCommuni{Honéstum fecit illum.}

\parsLectio{v}

\lettrine{I}{ndício} iniquíssimi proditóris detéctus,
mánibus post tergum revínctis Londínium addúcitur
abreptúsque in cárcerem conféstim in equúleum tóllitur,
atque ádeo crudéliter torquétur,
ut quassáto córpore paene semivívus jacéret.
Postrémo, licet nullo legítimo judício convíctus,
damnátur crudelíssimo supplícii génere,
quo perduelliónis rei plecti in Anglia solébant,
et impósitus crate vimínea ad Tybúrnum,
supplícii locum infámem, raptátur.

\parsLectio{vi}

\lettrine{E}{} crate in plaustrum subjéctum patíbulo sublátus
insertóque in láqueum collo,
circumfúsam úndique multitúdinem allocútus est,
seque cathólicum sacerdótem proféssus,
mortem pro fídei defensióne optatíssimam ultro se obíre affirmávit;
quandóquidem nullum sibi neque proditiónis neque conspiratiónis
crimen in regínam aut in pátriam óbjici posset.
Martýrium fecit Kaléndis Decémbris
anno millésimo quingentésimo octogésimo primo,
aetátis suae quadragésimo secúndo.
Cujus mártyris ejúsque sociórum cultum Gregórii décimi tértii
auctoritáte propósitum probatúmque,
Leo décimus tértius e senténtia sacrae Rituum Congregatiónis
sollémni décreto confirmávit quinto Idus Decémbris
anno millésimo octingentésimo octogésimo sexto.

\horaNocturnusIII

Léctio sancti Evangélii secúndum Matthaéum.

\parsLectioEv{Cap. 10, 34-42}

\lettrine{I}{n} illo témpore:
Dixit Jesus discípulis suis:
Nolíte arbitrári, quia pacem vénerim míttere in terram:
non veni pacem míttere sed gládium.
Et réliqua.

\parsHomilia{sancti Hilárii Epíscopi}{Can. 10}

\lettrine{Q}{uae} ista divísio est?
inter prima enim legis praecépta accépimus:
Honóra patrem tuum et matrem tuam;
et ipse Dóminus ait:
Pacem meam do vobis, pacem meam relínquo vobis.
Quid sibi vult missus pótius gládius in terram,
et separátus a patre fílius, et fília a matre,
et nurus advérsus socrum, et hóminis doméstici ejus inimíci?
Igitur exínde pública auctóritas impietáti proferétur.
Ubíque ódia, ubíque bella et gládius Dómini inter patrem et fílium,
et inter fíliam matrémque desaéviens.

\rubRespDeCommuni{Coróna áurea.}

\parsLectio{viii}

\lettrine{G}{ládius} telórum ómnium telum acutíssimum est,
in quo sit jus potestátis, et judícii sevéritas,
et animadvérsio peccatórum.
Et hujus quidem teli nómine novi Evangélii praedicatiónem
appellátam frequens in Prophétis auctóritas est.
Dei igitur verbum nuncupátum meminérimus in gládio:
qui gládius missus in terram est, idest,
praedicátio ejus hóminum córdibus infúsa.
Fitque gravis in domo una dissénsio,
et doméstica novo hómini erunt inimíca:
quia ille per verbum Dei divísus ab illis,
manére intérior et extérior, idest,
et corpus et ánima, in spíritus novitáte gaudébit.

\parsLectio{ix}

\lettrine{P}{ergit} deínde eódem praeceptórum et intellegéntiae
decúrsu. Nam, posteáquam relinquénda ómnia,
quae in saéculo caríssima sunt, imperáverat, adjécit:
Qui non áccipit crucem suam, et séquitur me, non est me dignus:
quia Qui Christi sunt, crucifixérunt corpus cum vítiis
et concupiscéntia.
Et indígnus est Christo, qui non crucem suam, in qua compátimur,
commórimur, consepelímur, conresúrgimus, accípiens,
Dóminum sit secútus, in hoc sacraménto fídei spíritus novitáte
victúrus.

\parsTeDeum

\rubrica{Vesperae a capitulo de sequenti, commemoratio praecedentis:
  antiphona}
Qui vult veníre post me.
\rubrica{versus}
Justus ut palma.
\rubrica{Oratio ut supra. Deinde fit Com. feriae.}

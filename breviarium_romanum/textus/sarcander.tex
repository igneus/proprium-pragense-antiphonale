\begin{caputFesti}
  \diesFesti{Die 17 Martii}
  \nomenFesti{B. Joannis Sarcander}
  \descriptioFesti{Martyris}
  \dignitasFesti{Duplex}
\end{caputFesti}

\parsOratio
Deus, qui beátum Joánnem Mártyrem tuum
in confessióne verae fídei
et sacramentális siléntii custódia
virtúte constántiae roborásti:
praesta, quaésumus;
ut contra advérsa ómnia ejus muniámur exémplis,
et protegámur auxíliis.
Per Dóminum.

\rubrica{Et fit Commemoratio S. Patricii Episcopi et Confessoris
  per Antiphonam}
Sacérdos et Póntifex.
\rubrica{et Versum}
Amávit eum.

\parsOratio
Deus, qui ad praedicándam géntibus glóriam tuam
beátum Patrícium Confessórem atque Pontíficem
míttere dignátus es:
ejus méritis et intercessióne concéde;
ut, quae nobis agénda praécipis,
te miseránte, adimplére possímus.
(Per Dóminum.)

\rubrica{Deinde fit Commemoratio Feriae.}

\rubrica{In I Nocturno Lectiones}
Fratres: Debitóres,
\rubrica{de Communi plurimorum Martyrum cum Responsoriis unius Martyris.}

\horaNocturnusII

\parsLectio{iv}

\lettrine{J}{oánnes} Sarcánder,
Skočóviae in ducátu Teschinénsi superióris Silésiae nóbili génere ortus,
adhuc puer, amísso patre, tutélae matérnae créditus,
infensíssima cathólico nómini tempestáte,
piam in lítteris et móribus institutiónem quaéritans,
in Moráviam finitimásque regiónes conténdit.
Freibérgae primum consístit, deínde Olomúcium,
mox Pragam in Bohémia, postrémo Graécium in Stýria
ad humanióres et sacras disciplínas ediscéndas proficíscitur.
Sacerdótio itémque doctóris título auctus,
in Moráviam dénuo revérsus,
a Cardináli Epíscopo Olomucénsi cúriae Holešoviénsi post grassátam
ibi octogínta annos haéresim praefícitur.
Pastorále munus rite fidelitérque gessit;
verbo et exémplo pietáti fovéndae et amplificándae,
aberrántibus ad cathólicam unitátem revocándis ac recipiéndis,
ecclesiásticae immunitátis júribus tuéndis,
orthodóxis doctrínis de sacraméntis Paeniténtiae et Eucharístiae,
a sacrosáncto Concílio Tridentíno tráditis ac confirmátis,
ácriter propugnándis, óperam sédulo navávit;
ex quo ómnium haereticórum invídias et simultátes in se convértit,
Bohémica seditióne gliscénte, tantísper Holešóvio migrávit,
et in Polóniam ad beátam Maríam Vírginem Czenstochóvii
ex voto secéssit:
sed morae impátiens, ac desidério fidélium in fide confirmandórum
incénsus, cum circa Silesiános fines diu oberrásset,
íterum Holešóvium se recépit.
Proscríptus brevi, et in jus vocátus,
in arce Tovačóvii se subrípuit,
donec in silva Olomúcio próxima próditus,
et a satellítibus abréptus,
contumélias et calúmnias haereticórum mira ánimi constántia
ac mansuetúdine perpéssus, in víncula conjéctus est.

\rubRespDeCommuni{Honéstum fecit illum.}

\parsLectio{v}

\lettrine{I}{nstitúta} inquisitióne, quater in judícium addúctus,
coram saevíssimis optimátibus Moravórum convíciis,
maledíctis et exsecratiónibus excéptus:
ter, suspénsus in equúleo, arctíssime, ad sex continéntes horas,
attráctus extentúsque est:
bis vero admótis ad látera per quinque horas fácibus sebo,
súlphure, resína imbútis:
tandem adhíbitis plumis, picis ac ólei liquámine íllitis et accénsis,
inque látera, ventrem, collum, axillásque impáctis,
ita adústus est, ut, depástis flamma cárnibus,
vix víscera inter costárum repágula cohiberéntur.
Hisce cruciátibus in Mártyrem animadvértunt haerétici
tum propter ódium disciplínae ac fídei cathólicae,
eas in regiónes invéctae, ac strénue propugnátae,
tum ut dynástae de Lóbkovic arcána,
in exomologési pándita, de cónscii sacerdótis péctore elícerent
atque extorquérent.
Frustra diu invícti Mártyris contra honórem sacraménti tentáta
constántia est:
nam patíbulum patiéntis factum est cáthedra docéntis,
ex qua protestatiónibus, adhortatiónibus et eníxis ad Deum précibus,
necnon jugi et validíssima sanctórum nóminum Jesu, Maríae et Annae
invocatióne, júdicum rábiem devícit et feritátem carníficum prostrávit.

\parsLectio{vi}

\lettrine{L}{ictóri} haerético et cárceri perpétuo tráditus,
in oratióne ac caeléstium rerum contemplatióne relíquum passiónis
suae tempus,
quod inter acerbíssimos dolóres ad trigínta tres dies perdúctum est,
insúmpsit.
Cotídie Horas Canónicas recitávit:
cumque ob diffráctos nervos et disrúptos totíus córporis artus
impos ad múnia vitae obeúnda redderétur,
linctu linguae opem mánuum in páginis verténdis supplébat.
Tandem die décima séptima Mártii
anni millésimi sexcentésimi vigésimi, annos natus quadragínta tres,
victrícem ánimam caelo inseréndam réddidit in illis verbis:
Convértere, ánima mea, in réquiem tuam,
quia Dóminus benefécit tibi:
quia erípuit ánimam meam de morte, óculos meos a lácrimis,
pedes meos a lapsu.
Ejus corpus ob inédiam et diutúrnam tolerántiam,
viscerúmque putrefactiónem foetens ac squálidum,
post mortem quodámmodo juvénta refloruísse,
et suávem odórem exhalásse compértum est,
ita ut illud Cathólici ad septem dies inhumátum serváverint.
Quod demum in Ecclésia Deíparae Vírginis sidéribus recéptae
Olomúcii cónditum est.
Antrum supplícii in Ecclésiam convérsum est;
ibi equúleus servátur septo conclúsus,
ne importúna fidélium pietáte in frusta comminuátur;
ibíque fons aquae perénnis vísitur,
qui ad levándam Mártyris sitim súbito scátuit,
valde céleber, quod ejus haustu febres fugántur.
Quem miráculis clarum Pius nonus Póntifex Máximus
anno millésimo octingentésimo sexagésimo
beatórum Caélitum albo accénsuit,
et inter cathólicae Ecclésiae mártyres réttulit.

\rubrica{In III Nocturno Homilia in Evangelium}
Nihil est opértum,
\rubrica{de Communi unius Martyris, 4 loco.}

\rubrica{IX Lectio de Homilia Feriae.}

\rubrica{In Laudibus fit Commemoratio S. Patricii Episcopi et
  Confessoris per Antiphonam}
Euge, serve bone.
\rubrica{et Versum}
Justum dedúxit.
\rubrica{Oratio ut supra.}

\rubrica{Deinde fit Commemoratio Feriae.}

\rubrica{Vesperae a Capitulo de sequenti,
  Commemoratio praecedentis per Antiphonam}
Qui vult veníre.
\rubrica{et Versum}
Justus ut palma.
\rubrica{Oratio ut supra.}

\rubrica{Deínde Commemoratio S. Patricii Episcopi et Confessoris
  per Antiphonam}
Amávit eum.
\rubrica{et Versum}
Justum dedúxit.
\rubrica{Oratio ut supra.}

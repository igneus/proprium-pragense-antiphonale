\begin{caputFesti}
  \diesFesti{Die 2 Martii}
  \nomenFesti{B. Agnetis de Bohemia}
  \descriptioFesti{Virginis Ordinis Clarissarum}
  \dignitasFesti{Duplex}

  [Nota: Anno 1989 a Ioanne Paulo papa II est canonizata.]
\end{caputFesti}

\parsOratio

Deus, qui beátam Agnétem Vírginem
per regálium deliciárum contémptum
et húmilem tuae crucis sequélam
ad caelum sublimásti:
tríbue nobis, quaésumus,
ut ejus précibus et imitatióne
aetérnae glóriae mereámur esse partícipes.
Qui vivis.

\horaNocturnusII

\parsLectio{iv}

\lettrine{A}{gnes} Přemyslái Ottokári primi regis Bohemórum fília,
Pragae in pervigílio sanctae Vírginis et Mártyris Agnétis
in lucem édita fuit.
Bolesláo Silésiae ducis fílio deínde desponsáta a paréntibus,
et in monastério Trebnicénsi vírginum Cisterciénsium
prope Vratisláviam,
curánte beáta Hedwíge ducíssa collocáta,
ibi prima sanctitátis fundaménta jecit.
Cum triénnium in eo exegísset, defúncto sponso,
Dóxanam ad sanctimoniáles Praemonstraténses migrávit,
inter quas omni virtútum génere pro ténerae aetátis módulo enítuit.
Aliórum quoque núptias, non modo regum, sed et imperatóris,
non semel póstea oblátas constánter declinávit,
et ad Príncipem Austriae, áulicis erudiénda consuetudínibus missa,
non terréni sed caeléstis regni gustándis ánimum virésque omnes inténdit.
Indeféssa pérmanens in oratióne,
carnem suam jejúniis atque áliis asperitátibus edómuit;
erga páuperes vero et oppréssos téneram commiseratiónem aeque
ac profúsam liberalitátem exhíbuit.
Eándem vitae ratiónem ferventióri stúdio servávit in patérna domo,
cum quólibet sponsálium vínculo solúta esset;
donec, Summo Pontífice Gregório nono adjuvánte,
a régio fratre suo líbere sequéndi divínum sponsum Jesum Christum
et coenóbium ingrediéndi facultátem impetrávit.

\rubRespDeCommuni{Propter veritátem.}

\parsLectio{v}

\lettrine{A}{ttígerat} illis diébus Bohémiam fama novi reguláris
institúti a beáta Clara Assisiénsi cónditi:
qua permóta íllico Agnes mundánis omníno valedícere próperat.
Amplum paupéribus excoléndis in urbe Pragéna hospítium éxstruit,
recentíque Crucigenórum cum stella rúbea nuncupatórum órdini
regéndum commíttit.
Primum dein eádem in urbe Clarissárum monastérium érigit,
beátae ipsíus Agnétis nómine póstea insignítum;
illúdque tradit sanctimoniálibus virgínibus,
quas proptérea ab eádem institutríce Clara ad se mitti postuláverat.
Hisce mox et ipsa sociáta hábitu et sacro velámine ibi assúmpto,
álias nobilíssimas vírgines secum addúxit
ad novum severióris vitae institútum amplecténdum.
Ab ipso autem Gregório nono deínceps monastérii Abbatíssa constitúta
religiósam famíliam sanctíssime gubernávit.
Quamvis enim esset ómnium prima,
non áliter céteris praeésse visa est,
nisi praecláro devotiónis, oboediéntiae, castitátis,l
sui abnegatiónis ac demissiónis exémplo.

\parsLectio{vi}

\lettrine{S}{ed} vírginem humilitátis amantíssimam
honórum véluti postrémae relíquiae perturbábant.
Quare, brevi transácto témpore, Abbatíssae títulum recusávit,
ac tantúmmodo soror major monastérii deínceps vóluit appellári.
Oblátas a régio fratre divítias accípere rénuit,
nullúmque a soróribus retinéri permísit temporálium bonórum domínium.
Supernárum visiónum dono et grátia curatiónum ditáta a Deo
fuísse perhibétur.
Jamque illi soli vivens, cum in ómnium virtútum cultúra constánter
perseverásset,
tandem, cumuláta méritis, in caelum migrávit circa reparátae salútis
annum ducentésimum octogésimum supra millésimum,
aetáte fere octogenária.
Cultum ab immemorábili témpore beátae Agnéti praéstitum
Pius nonus Póntifex Máximus, ex Sacrórum Rítuum Congregatiónis consúlto,
apostólica auctoritáte ratum hábuit et confirmávit.

\rubrica{In III Nocturno Homilia in Evangelium}
Símile erit regnum caelórum,
\rubrica{de Communi Virginum 1 loco.}

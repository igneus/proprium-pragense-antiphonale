\begin{caputFesti}
  \diesFesti{Die 22 Decembris}
  \dioecBud
  \nomenFesti{In Dedicatione Ecclesiae Cathedralis}
  \dignitasFesti{Duplex I classis cum Octava communi,
    de qua, ratione temporis, fit tantum die Octava}
\end{caputFesti}

\rubrica{In I Vesperis
  Commemoratio S. Thomae Apostoli
  et Feriae}
O Oriens.

\rubrica{Omnia ut in Communi Dedicationis Ecclesiae.
  In Laudibus Commemoratio Feriae.}

\rubrica{In II Vesperis Commemoratio Feriae}
O Rex.

\begin{caputFesti}
  \diesFesti{Die 29 Decembris}
  \dioecBud
  \nomenFesti{In Octava Dedicationis Ecclesiae Cathedralis}
  \dignitasFesti{Duplex majus}
\end{caputFesti}

\rubrica{In II Vesperis Ss. Innocentium Martyrum
  Commemoratio Octavae Dedicationis e I Vesperis
  ut in Communi Dedicationis;
  deinde Commemoratio S. Thomae Episcopi et Martyris
  et Octavae Nativitatis.}

\rubrica{Lectiones II et III Nocturni ut in Communi Dedicationis
  Ecclesiae die Octava.}

\rubrica{Pro S. Thoma Episcopo et Martyre:}

\parsLectio{ix}

Thomas, Londíni in Anglia natus,
ántea regni cancellárius, Theobáldo succéssit Cantuarénsi epíscopo.
In episcopáli offício fortis et invíctus,
leges utilitáti ac dignitáti ecclesiásticae repugnántes,
ab Henríco secúndo rege latas,
nullis fractus suis ac suórum incómmodis,
acceptáre rénuit.
Quare próxime conjiciéndus in cárcerem, clam recéssit;
et primo Pontiníaci apud mónachos Cisterciénses,
deínde apud Ludovícum regem Gálliae se cóntulit.
Ab exsílio revocátus, paulo post calúmniam apud regem ita impétitur,
ut saépius conquererétur rex,
se in suo regno cum uno sacerdóte pacem habére non posse.
Hinc nefárii hómines, sperántes se gratum regi factúros,
Thomam in Cantuariénsi templo vespertínis horis óperam dantem aggrediúntur.
Qui cléricis templi áditus praeclúdere conántibus óbstitit, dicens:
Non est Dei ecclésia custodiénda more castrórum;
et ego pro Ecclésia Dei libénter mortem subíbo.
Tum ad mílites ait: Vos Dei jussu cavéte,
ne cuípiam meórum noceátis.
Deínde, flexis génibus, ecclésiam et seípsum Deo comméndans,
cápite pléctitur,
quarto Kaléndas Januárii, anno Dómini millésimo centésimo
septuagésimo primo.

\parsTeDeum

\rubrica{Ad Laudes fit Commemoratio S. Thomae Episcopi et Martyris
  et Octavae Nativitatis.}

\rubrica{In II Vesperis Commemoratio sequentis diei infra Octavam
  Nativitatis
  et S. Thomae.}

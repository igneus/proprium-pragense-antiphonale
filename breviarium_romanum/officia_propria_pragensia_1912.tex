\documentclass[12pt, a5paper, twoside]{book}

% To compile:
%
% $ xelatex filename
% $ biber filename
% $ xelatex filename

\usepackage{fontspec}
\setmainfont[Ligatures={TeX}]{Junicode}

\usepackage[latin]{babel}
%\usepackage{csquotes}
\usepackage[left=2cm, right=1.5cm, top=1.5cm, bottom=1.5cm, marginparsep=3mm]{geometry}

\usepackage[
  backend=biber,
  style=authortitle,
  sortlocale=la_LA,
  maxnames=3,
  firstinits=true,
]{biblatex}

\usepackage[show]{ed} % editorial annotations
\usepackage[hidelinks]{hyperref}
\usepackage{multicol}
\usepackage{parcolumns}
\usepackage{lettrine}

\usepackage{fontspec}
\usepackage{etoolbox}
\usepackage{luatextra}
\usepackage{graphicx} % support the \includegraphics command and options
\usepackage{gregoriotex} % for gregorio score inclusion

\newenvironment{caputFesti}{\begin{center}}{\end{center}}

% pieces of the title of a feast
\newcommand{\diesFesti}[1]{{\small #1} \vspace{2mm}\\}
\newcommand{\nomenFesti}[1]{\textbf{\Large #1}\vspace{3mm}\\}
\newcommand{\descriptioFesti}[1]{\textbf{#1}\\}
\newcommand{\dignitasFesti}[1]{\small #1}

% headings of various parts
\newcommand{\hora}[1]{\vspace{5mm}\noindent\textbf{#1}\vspace{2mm}}
\newcommand{\horaVesperaeI}{\hora{In I Vesperis}}
\newcommand{\horaLaudes}{\hora{In Laudibus}}
\newcommand{\horaLaudesEtHorae}{\hora{Ad Laudes et per Horas}}
\newcommand{\horaTertia}{\hora{Ad Tertiam}}
\newcommand{\horaSexta}{\hora{Ad Sextam}}
\newcommand{\horaNona}{\hora{Ad Nonam}}
\newcommand{\horaVesperaeII}{\hora{In II Vesperis}}

\newcommand{\parsHorae}[1]{\textbf{#1}\vspace{1mm}}
\newcommand{\parsOratio}{\begin{center}\emph{Oratio}\end{center}}
\newcommand{\parsCapitulum}[1]{\hspace{1.2cm}\textsc{Capitulum.}\hfill #1\hspace{1cm}}

\newcommand{\rubrica}[1]{{\small \emph{#1}}}

\newcommand{\utInBreviarioPraeter}{%
  \rubrica{Omnia ut in Antiphonario sub hac die, praeter sequentia.}}
\newcommand{\perDominum}{Per Dóminum.}

\newcommand{\versiculus}[2]{\noindent ℣. #1\\℟. #2}

% calendar
\newcommand{\calMonth}[1]{\begin{center}\textbf{#1}\end{center}}


\bibliography{bibliography}

\setcounter{secnumdepth}{3}
\setcounter{tocdepth}{3}

\title{Officia propria ecclesiasticae provinciae Pragensis}

\begin{document}

\pagestyle{empty}

\setlength{\parindent}{0.5cm}

\maketitle

\cleardoublepage

\pagestyle{plain}

\chapter*{Ratio huius editionis}

Breviarium proprium ecclesiae Pragensis ter
(1502, 1509, 1517)\footcite[242]{bohatta}
typis impressum est; post Breviarii Romani receptionem
saeculo XVII unumquidque saeculum plurimis editionibus
libelli \emph{Officia propria ecclesiasticae provinciae Pragensis} gaudebat.
Sed notae musicae usque ad dies nostros numquam typis sunt impressae
et tantum magnis cum difficultatibus in libris manu scriptis
inveniri possunt. Ideo nos antiphonalia vetustissima recentioresque
invenimus et comparavimus, et quos in illis invenimus,
cantus hic offere audemus.

\section*{De textibus et calendario}
Calendarium atque textus officiorum sumuntur de editionibus
\emph{Officiorum propriorum ecclesiasticae provinciae Pragensis}
anno 1912 et ultra editis.
Ubi post annum 1912 festum novum indultum, promotum, demotum
aut officium vetus mutatum est, hoc semper annotatum est,
cum anno quo mutatio haec occurrit.

Quia post annum 1960, quo Calendarium Breviariumque Romanum
graviter reformata sunt, nova \emph{Officia propria Pragensia}
pro adversitate temporis non sunt edita,
in secunda parte libri officia propria huic novo ordine Breviarii
accommodata proponimus. Sed nota quod propositio haec nullam habet
ecclesiasticam approbationem.

\section*{De fontibus musicis adhibitis}
Officia nonnulla, ex. g. S. Viti, S. Adalberti, Ss. Cosmae et Damiani,
etiam post omnes instaurationes mutationesque, quibus
Breviarii proprium Pragense a saeculo XVII usque ad dies nostros
subiectum est, ex toto aut ex parte textus vetustissimos
continent.
Eorum melodiae de antiphonariis antiquis sumpsimus,
ut pro unaquaque antiphona et hymno proprio loco annotatur.
Quia melodiae tradendo variabantur et versio recentior nonnumquam
pulchrior et cantabilior invenitur, non semper versionem
vetustissimam reddimus.
(Quis illam maxime authenticam et restituendam putat,
in libris doctorum inveniet.)

Officia permulta aut nova aetate sunt indulta
(ut S. Ioannis Nepomuceni),
aut amplificata,
aut graviter reformata (ut S. Ludmilae, S. Venceslai, S. Procopii).
Ubi potuimus, ea in libris retentioribus, aut impressis
(pro S. Ioanne Nepomuceno et partim pro Ss. Cyrillo et Methodio),
aut manu scriptis invenimus et inventa reddimus.

\section*{De cantibus accommodandis}
A saeculo XVII unaquaeque fere editio \emph{Officiorum propriorum}
mutationes aliquas in officia vetera introduxit.
Permultae antiphonae vetustissimae abbreviatae sunt aut
alio modo immutatae.
In melodiis ad textus immutatos accommodandis semper studuimus
maximam partem melodiae veteris praeservare.
De omnibus immutationibus proprio loco rationem reddimus.

\section*{De cantibus componendis}
Officia nonnulla, praesertim saeculo XIX composita et indulta,
notis musicis numquam instructa sunt.
Pro his modulationes proprias composuimus, in stylo cantus gregoriani
et/aut cantuum veterorum Ecclesiae Pragensis.
Ubi melodia propria nondum composita est, textum saltem reddimus.

\section*{De libri huius partibus}
Liber hic quatuor habet partes.
Pars prima continet omnia officia hoc numero et ordine disposita
ut anno 1912, cum Antiphonale novum restitutum
cum dispositione nova psalterii in lucem editum est.
Pars secunda continet officia nova, amplificata aut mutata,
sed etiam praescribit, quomodo alia officia cum
Calendario Romano anno 1960 reformato adhibenda sunt.
Quia nonnullae antiphonae aut officia tota plurima cantus versione
gaudent, pars tertia continet versiones, quae loco eorum,
quae prima parte impressae sunt, ad libitum adhiberi possunt.
In quarta parte praesertim de antiphonis responsoriisque,
quos ipsi notis musicis adornavimus, rationem reddimus.


\part{Proprium Pragense ad~Antiphonale Sacrosanctae Romanae Ecclesiae
  a Pio Papa~X anno 1912 editum}

\section*{Kalendarium perpetuum Provinciae Pragenae\\1915+1928}

{\footnotesize
  Fontes adhibiti:
  \emph{Officia propria: pars hiemalis,} Kotrba Pragae 1915.
  \emph{Officia propria: pars autumnalis,} Kotrba Pragae 1915.
  \emph{Officia propria: pars verna,} Pustet Ratisbonae 1928.
  \emph{Officia propria: pars aestiva,} Pustet Ratisbonae 1928.
}

\input{calendarium/1915}
\cleardoublepage

\section*{Kalendarium proprium Provinciae Pragenae\\1949}

{\footnotesize
  Kalendarium sequens tantum festa propria provinciae Pragenae
  continet.

  Fontes adhibiti:
  \emph{Officia propria: pars hiemalis,} Gottmer Harlemi 1949.
  \emph{Officia propria: pars autumnalis,} Gottmer Harlemi 1949.
  \emph{Officia propria: pars verna,} Gottmer Harlemi 1949.
  (Parte aestivali, quam usque ad hunc diem locare non potuimus,
  kalendarium caret.)
}

\input{calendarium/1947}
\cleardoublepage

% incipitur a parte hiemali

\begin{caputFesti}
  \diesFesti{Die 1. Decembris}
  \dioecPraga
  \nomenFesti{B. Edmundi Campion}
  \descriptioFesti{Martyris e S. J.}
  \dignitasFesti{Semiduplex}
\end{caputFesti}

\rubrica{In II Vesperis praecedentis fit Com.
  per antiphonam}
Iste sanctus pro lege.
\rubrica{et versum}
Glória et honóre.

\parsOratio
Deus, qui verae fídei
et Sedis Apostólicae primátui propugnándo
beátum Mártyrem tuum Edmúndum
invícta fortitúdine roborásti:
ejus précibus exorátus,
nostrae, quaésumus, infirmitáti succúrre;
ut fortes in fide adversário resístere usque in finem valeámus.
Per Dóminum.

\rubrica{Et fit commemoratio feriae.}

\horaNocturnusII

\parsLectio{iv}

\lettrine{E}{dmúndus} Cámpion Londínii in Anglia natus,
humanióribus lítteris in Oxoniénsi academía,
tum sacris disciplínis Duáci in Anglórum seminário óptime excúltus,
Romae demum Societáti Jesu nomen dedit,
Pragae inter novítios recéptus,
primus Congregatiónis Beátae Maríae Vírginis praeses eléctus est,
et sacerdótio auctus per duos annos sacros sermónes habébat.
In pátriam Summi Pontíficis jussu,
una cum Robérto Persons, revérsus,
paucis ménsibus ea perfécit exémplo vitae,
excellénti doctrína atque agéndi dexteritáte,
ut ómnium ad se ánimos convérterit,
catholicórum quidem, ut eum audírent servaréntque,
inimicórum ut pérderent.

\rubRespDeCommuni{Honéstum fecit illum.}

\parsLectio{v}

\lettrine{I}{ndício} iniquíssimi proditóris detéctus,
mánibus post tergum revínctis Londínium addúcitur
abreptúsque in cárcerem conféstim in equúleum tóllitur,
atque ádeo crudéliter torquétur,
ut quassáto córpore paene semivívus jacéret.
Postrémo, licet nullo legítimo judício convíctus,
damnátur crudelíssimo supplícii génere,
quo perduelliónis rei plecti in Anglia solébant,
et impósitus crate vimínea ad Tybúrnum,
supplícii locum infámem, raptátur.

\parsLectio{vi}

\lettrine{E}{} crate in plaustrum subjéctum patíbulo sublátus
insertóque in láqueum collo,
circumfúsam úndique multitúdinem allocútus est,
seque cathólicum sacerdótem proféssus,
mortem pro fídei defensióne optatíssimam ultro se obíre affirmávit;
quandóquidem nullum sibi neque proditiónis neque conspiratiónis
crimen in regínam aut in pátriam óbjici posset.
Martýrium fecit Kaléndis Decémbris
anno millésimo quingentésimo octogésimo primo,
aetátis suae quadragésimo secúndo.
Cujus mártyris ejúsque sociórum cultum Gregórii décimi tértii
auctoritáte propósitum probatúmque,
Leo décimus tértius e senténtia sacrae Rituum Congregatiónis
sollémni décreto confirmávit quinto Idus Decémbris
anno millésimo octingentésimo octogésimo sexto.

\horaNocturnusIII

\parsEvangelium{Matthaéum}

\parsLectioEv{Cap. 10, 34-42}

\lettrine{I}{n} illo témpore:
Dixit Jesus discípulis suis:
Nolíte arbitrári, quia pacem vénerim míttere in terram:
non veni pacem míttere sed gládium.
Et réliqua.

\parsHomilia{sancti Hilárii Epíscopi}{Can. 10}

\lettrine{Q}{uae} ista divísio est?
inter prima enim legis praecépta accépimus:
Honóra patrem tuum et matrem tuam;
et ipse Dóminus ait:
Pacem meam do vobis, pacem meam relínquo vobis.
Quid sibi vult missus pótius gládius in terram,
et separátus a patre fílius, et fília a matre,
et nurus advérsus socrum, et hóminis doméstici ejus inimíci?
Igitur exínde pública auctóritas impietáti proferétur.
Ubíque ódia, ubíque bella et gládius Dómini inter patrem et fílium,
et inter fíliam matrémque desaéviens.

\rubRespDeCommuni{Coróna áurea.}

\parsLectio{viii}

\lettrine{G}{ládius} telórum ómnium telum acutíssimum est,
in quo sit jus potestátis, et judícii sevéritas,
et animadvérsio peccatórum.
Et hujus quidem teli nómine novi Evangélii praedicatiónem
appellátam frequens in Prophétis auctóritas est.
Dei igitur verbum nuncupátum meminérimus in gládio:
qui gládius missus in terram est, idest,
praedicátio ejus hóminum córdibus infúsa.
Fitque gravis in domo una dissénsio,
et doméstica novo hómini erunt inimíca:
quia ille per verbum Dei divísus ab illis,
manére intérior et extérior, idest,
et corpus et ánima, in spíritus novitáte gaudébit.

\parsLectio{ix}

\lettrine{P}{ergit} deínde eódem praeceptórum et intellegéntiae
decúrsu. Nam, posteáquam relinquénda ómnia,
quae in saéculo caríssima sunt, imperáverat, adjécit:
Qui non áccipit crucem suam, et séquitur me, non est me dignus:
quia Qui Christi sunt, crucifixérunt corpus cum vítiis
et concupiscéntia.
Et indígnus est Christo, qui non crucem suam, in qua compátimur,
commórimur, consepelímur, conresúrgimus, accípiens,
Dóminum sit secútus, in hoc sacraménto fídei spíritus novitáte
victúrus.

\parsTeDeum

\rubrica{Vesperae a capitulo de sequenti, commemoratio praecedentis:
  antiphona}
Qui vult veníre post me.
\rubrica{versus}
Justus ut palma.
\rubrica{Oratio ut supra. Deinde fit Com. feriae.}

\begin{caputFesti}
  \diesFesti{Die 6 Decembris}
  \dioecBud
  \nomenFesti{S. Nicolai}
  \descriptioFesti{Episcopi et Confessoris, Titularis Ecclesiae Cathedralis}
  \dignitasFesti{Duplex 1 classis cum Octava communi}
\end{caputFesti}

\rubrica{Omnia etiam Lectiones 1 Nocturni}
Fidélis sermo,
\rubrica{de Communi Confessoris Pontificis,
  praeter ea, quae in Breviario habentur propria.}

\rubrica{In Laudibus Commemoratio Feriae; in II Vesperis
  Commemoratio sequentis et Feriae.}

\begin{caputFesti}
  \diesFesti{Die 13 Decembris}
  \dioecBud
  \nomenFesti{In Octava S. Nicolai}
  \descriptioFesti{Episcopi et Confessoris}
  \dignitasFesti{Duplex majus}
\end{caputFesti}

\rubrica{In I Vesperis fit Commemoratio
  Octavae Immaculatae Conceptionis
  et S. Luciae Virginis et Martyris et Feriae.}

\horaNocturnusII

\parsSermo{sancti Gregórii Papae}{Part. 2 Pastoralis, cap. 1}

\parsLectio{iv}

\lettrine{T}{antum} debet actiónem pópuli áctio transcéndere praésulis,
quantum distáre solet a grege vita pastóris.
Opórtet namque, ut metíri se sollícite stúdeat,
quanta tenéndae rectitúdinis necessitáte constríngitur,
sub cujus aestimatióne pópulus grex vocátur.
Sit ergo necésse est cogitatióne mundus,
actióne praecípuus,
discrétus in siléntio,
útilis in verbo,
síngulis compassióne próximus,
prae cunctis contemplatióne suspénsus,
bene agéntibus per humilitátem sócius,
contra delinquéntium vítia per zelum justítiae eréctus,
internórum curam in exteriórum occupatióne non mínuens,
exteriórum providéntiam in internórum sollicitúdine non relínquens.

\rubRespDeCommuni{Invéni David.}

\parsLectioAnnot{v}{Part. 1 Pastor. Cap. 9 et 10}

\lettrine{C}{onsiderándum} quoque est, quia cum curam pópuli eléctus
praésul súscipit, quasi ad aegrum médicus accédit.
Si ergo adhuc in ejus córpore passiónes vivunt,
qua praesumptióne percússum medéri próperat,
qui in fácie vulnus portat?
Ille ígitur modis ómnibus debet ad exémplum bene vivéndi pértrahi,
qui cunctis carnis passiónibus móriens,
jam spiritáliter vivit,
qui próspera mundi postpónit,
qui nulla advérsa pertiméscit,
qui sola intérna desíderat;
cujus intentióni bene cóngruens,
nec omníno per imbecillitátem corpus,
nec valde per contumáciam repúgnat spíritus:
qui ad aliéna cupiénda non dúcitur,
sed própria largítur.

\parsLectioAnnot{v}{Ibidem Capite 8}

\lettrine{U}{nde} ipsum quoque Episcopátus offícium boni óperis
expressióne definítur, cum dícitur:
Si quis Episcopátum desíderat, bonum opus desíderat.
Ipse ergo sibi testis est, quia Episcopátum non áppetit,
qui non per hunc boni óperis ministérium,
sed honóris glóriam quaerit.
Sacrum quippe offícium non solum non díligit omníno,
sed nescit, qui ad culmen regíminis anhélans,
in occúlta meditatióne cogitatiónis,
ceterórum subjectióne páscitur,
laude própria laetátur,
ad honórem cor élevat,
rerum affluéntium abundántia exsúltat.
Mundi ergo lucrum quaéritur sub ejus honóris spécie,
quo mundi déstrui lucra debúerant.


Officium olim plenum Bohemiae cantabatur,
de quo post receptionem Breviarii Romani duae solum antiphonae in proprium provinciale
sunt receptae.
Textus ambarum hac occasione abbreviatus fuit.\edissue{Check the exact history of edits, I haven't excerpted it before}


\vfill
\clearpage

Hymnos laudum vesperarumque, qui in proprium pragense ab editione
1643 accepti sunt, David Drachovský de Hornstein, archidiaconus Plsnensis
(ob. post 1624) scripsit.
Omnes cantus alii (etiam hymnus matutini) anno 1865 de novo sunt scripti,
historiam rythmicam vetustissimam \emph{Adest dies laetitiae}
totum eradicantes.


\part{Accommodationes pro Calendario Romano anno 1960 reformato}

\section*{Kalendarium proprium Provinciae Pragenae\\1960}

{\footnotesize
  Kalendarium sequens tantum festa propria provinciae Pragenae
  continet.

  Fontes adhibiti:
  \emph{Directorium 1962,} Česká katolická Charita v Ústředním církevním nakladatelství v Praze.
  \emph{Directorium 1964,} Česká katolická Charita v Ústředním církevním nakladatelství v Praze.
}

\begin{caputFesti}
  \diesFesti{Die 27 Septembris}
  \nomenFesti{Ss. Cosmae et Damiani}
  \descriptioFesti{Martyrum}
  \dignitasIII
\end{caputFesti}

\horaLaudes

\rubrica{Versiculus}
Exsultábunt sancti.
\rubrica{Ad Benedictus antiphona}
Beáti Mártyres cum duceréntur, \pageref{cosma:laudes}.
\rubrica{Oratio}
Magníficet te.

\begin{caputFesti}
  \diesFesti{Die 28 Septembris}
  \nomenFesti{S. Venceslai}
  \descriptioFesti{Ducis et Martyris, Patroni principalis Bohemiae}
  \dignitasI
\end{caputFesti}

\rubrica{Omnia ut supra, \pageref{venceslaus:vesperae1},
  tantum Commemoratio Octavae Dedicationis Ecclesiae
  Cathedralis Litomericensis non fit (festum Dedicationis
  non habet Octavam in Calendario reformato)
  et in I Vesperis fit Commemoratio Ss. Cosmae et Damiani.
  Antiphona}
Beáti Mártyres Christi.
\rubrica{Versiculus}
Laetámini in Dómino.
\rubrica{Oratio ut in festo, \pageref{cosma:vesperae1}.}

\cleardoublepage

\begin{caputFesti}
  \diesFesti{Die 27 Septembris}
  \nomenFesti{Ss. Cosmae et Damiani}
  \descriptioFesti{Martyrum}
  \dignitasIII
\end{caputFesti}

\horaLaudes

\rubrica{Versiculus}
Exsultábunt sancti.
\rubrica{Ad Benedictus antiphona}
Beáti Mártyres cum duceréntur, \pageref{cosma:laudes}.
\rubrica{Oratio}
Magníficet te.

\begin{caputFesti}
  \diesFesti{Die 28 Septembris}
  \nomenFesti{S. Venceslai}
  \descriptioFesti{Ducis et Martyris, Patroni principalis Bohemiae}
  \dignitasI
\end{caputFesti}

\rubrica{Omnia ut supra, \pageref{venceslaus:vesperae1},
  tantum Commemoratio Octavae Dedicationis Ecclesiae
  Cathedralis Litomericensis non fit (festum Dedicationis
  non habet Octavam in Calendario reformato)
  et in I Vesperis fit Commemoratio Ss. Cosmae et Damiani.
  Antiphona}
Beáti Mártyres Christi.
\rubrica{Versiculus}
Laetámini in Dómino.
\rubrica{Oratio ut in festo, \pageref{cosma:vesperae1}.}


%\part{Aliae cantuum versiones ad libitum adhibendae}

\part{Annotationes}

\input{annotationes}

%\part{Indices}

\printbibliography

\end{document}

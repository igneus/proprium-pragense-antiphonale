\documentclass[12pt, a5paper, twoside]{book}

% To compile:
%
% $ xelatex filename
% $ biber filename
% $ xelatex filename

\usepackage{fontspec}
\setmainfont[Ligatures={TeX}]{TeX Gyre Pagella}

\usepackage[latin]{babel}
%\usepackage{csquotes}
%\usepackage[left=2.5cm, right=2.5cm, top=2.5cm, bottom=2.5cm, marginparsep=3mm]{geometry}

%% \usepackage[
%%   backend=biber,
%%   style=iso-authoryear,
%%   sortlocale=cs_CZ,
%%   maxnames=3,
%%   firstinits=true,
%% ]{biblatex}

\usepackage[show]{ed} % editorial annotations
\usepackage{color}
\usepackage{xcolor}
\usepackage[hidelinks]{hyperref}
\usepackage{changepage}
\usepackage{filecontents}


\setcounter{secnumdepth}{3}
\setcounter{tocdepth}{3}

\title{Officia propria ecclesiasticae provinciae Pragensis}

\begin{document}

\setlength{\parindent}{0.5cm}

\maketitle

\chapter*{Ratio huius editionis}

Breviarium proprium ecclesiae Pragensis bis\edissue{annos supplere}
typis impressum est; post Breviarii Romani receptionem
saeculo XVII. unumquidque saeculum plurimas editiones
libelli Officia propria ecclesiasticae provinciae Pragensis vidit.
Sed notae musicae usque ad dies nostros numquam typis sunt impressae
et tantum magnis cum difficultatibus in libris manu scriptis
inveniri possunt. Ideo nos antiphonalia vetustissima retentioresque
invenimus et comparavimus, et quos invenimus, cantus hic offere
audemus.

Quia antiphonae nonnullae diebus retentioribus mutatae aut de novo
introductae, immo officia nova nonnulla composita sunt,
sed pro officii divini in choro cum cantu persoluti desuetudinem
cantus saepe accommodati aut compositi non sunt,
aliquos cantus ipsi composuimus aut accommodavimus, pro alios
textus tantum ponimus.

Liber hic quatuor habet partes.
Pars prima continet omnia officia hoc numero et ordine disposita
ut anno 1912, cum Antiphonale novum restitutum
cum dispositione nova psalterii in lucem editum est.
Pars secunda continet officia nova, amplificata aut mutata,
sed etiam praescribit, quomodo alia officia cum
Calendario Romano anno 1960 reformato adhibenda sunt.
Quia nonnullae antiphonae aut officia tota plurima cantus versione
gaudent, pars tertia continet versiones, quae loco eorum,
quae prima parte impressae sunt, ad libitum adhiberi possunt.
In quarta parte praesertim de antiphonis responsoriisque,
quos ipsi notis musicis adornavimus, rationem reddimus.

\part{Proprium Pragense ad Antiphonale Sacrosanctae Romanae Ecclesiae
  a Pio Papa X. anno 1912 editum}

\part{Accommodationes pro Calendario Romano anno 1960 reformato}

\part{Aliae cantuum versiones ad libitum adhibendae}

\part{Annotationes}

\part{Indices}

%\printbibliography

\end{document}

\documentclass[12pt, a5paper, twoside]{book}

% To compile:
%
% $ xelatex filename
% $ biber filename
% $ xelatex filename

\usepackage{fontspec}
\setmainfont[Ligatures={TeX}]{Junicode}

\usepackage[latin]{babel}
%\usepackage{csquotes}
\usepackage[left=2cm, right=1.5cm, top=1.5cm, bottom=1.5cm, marginparsep=3mm]{geometry}

\usepackage[
  backend=biber,
  style=authortitle,
  sortlocale=la_LA,
  maxnames=3,
  firstinits=true,
]{biblatex}

\usepackage[show]{ed} % editorial annotations
\usepackage[hidelinks]{hyperref}
\usepackage{multicol}
\usepackage{parcolumns}
\usepackage{lettrine}

\usepackage{fontspec}
\usepackage{etoolbox}
\usepackage{luatextra}
\usepackage{graphicx} % support the \includegraphics command and options
\usepackage{gregoriotex} % for gregorio score inclusion

\newenvironment{caputFesti}{\begin{center}}{\end{center}}

% pieces of the title of a feast
\newcommand{\diesFesti}[1]{{\small #1} \vspace{2mm}\\}
\newcommand{\nomenFesti}[1]{\textbf{\Large #1}\vspace{3mm}\\}
\newcommand{\descriptioFesti}[1]{\textbf{#1}\\}
\newcommand{\dignitasFesti}[1]{\small #1}

% headings of various parts
\newcommand{\hora}[1]{\vspace{5mm}\noindent\textbf{#1}\vspace{2mm}}
\newcommand{\horaVesperaeI}{\hora{In I Vesperis}}
\newcommand{\horaLaudes}{\hora{In Laudibus}}
\newcommand{\horaLaudesEtHorae}{\hora{Ad Laudes et per Horas}}
\newcommand{\horaTertia}{\hora{Ad Tertiam}}
\newcommand{\horaSexta}{\hora{Ad Sextam}}
\newcommand{\horaNona}{\hora{Ad Nonam}}
\newcommand{\horaVesperaeII}{\hora{In II Vesperis}}

\newcommand{\parsHorae}[1]{\textbf{#1}\vspace{1mm}}
\newcommand{\parsOratio}{\begin{center}\emph{Oratio}\end{center}}
\newcommand{\parsCapitulum}[1]{\hspace{1.2cm}\textsc{Capitulum.}\hfill #1\hspace{1cm}}

\newcommand{\rubrica}[1]{{\small \emph{#1}}}

\newcommand{\utInBreviarioPraeter}{%
  \rubrica{Omnia ut in Antiphonario sub hac die, praeter sequentia.}}
\newcommand{\perDominum}{Per Dóminum.}

\newcommand{\versiculus}[2]{\noindent ℣. #1\\℟. #2}

% calendar
\newcommand{\calMonth}[1]{\begin{center}\textbf{#1}\end{center}}


\bibliography{bibliography}

\setcounter{secnumdepth}{3}
\setcounter{tocdepth}{3}

\title{Officia propria ecclesiasticae provinciae Pragensis}

\begin{document}

\pagestyle{empty}

\setlength{\parindent}{0.5cm}

\maketitle

\cleardoublepage

\pagestyle{plain}

\chapter*{Ratio huius editionis}

Breviarium proprium ecclesiae Pragensis ter
(1502, 1509, 1517)\footcite[242]{bohatta}
typis impressum est; post Breviarii Romani receptionem
saeculo XVII unumquidque saeculum plurimis editionibus
libelli \emph{Officia propria ecclesiasticae provinciae Pragensis} gaudebat.
Sed notae musicae usque ad dies nostros numquam typis sunt impressae
et tantum magnis cum difficultatibus in libris manu scriptis
inveniri possunt. Ideo nos antiphonalia vetustissima recentioresque
invenimus et comparavimus, et quos in illis invenimus,
cantus hic offere audemus.

\section*{De textibus et calendario}
Calendarium atque textus officiorum sumuntur de editionibus
\emph{Officiorum propriorum ecclesiasticae provinciae Pragensis}
anno 1912 et ultra editis.
Ubi post annum 1912 festum novum indultum, promotum, demotum
aut officium vetus mutatum est, hoc semper annotatum est,
cum anno quo mutatio haec occurrit.

Quia post annum 1960, quo Calendarium Breviariumque Romanum
graviter reformata sunt, nova \emph{Officia propria Pragensia}
pro adversitate temporis non sunt edita,
in secunda parte libri officia propria huic novo ordine Breviarii
accommodata proponimus. Sed nota quod propositio haec nullam habet
ecclesiasticam approbationem.

\section*{De fontibus musicis adhibitis}
Officia nonnulla, ex. g. S. Viti, S. Adalberti, Ss. Cosmae et Damiani,
etiam post omnes instaurationes mutationesque, quibus
Breviarii proprium Pragense a saeculo XVII usque ad dies nostros
subiectum est, ex toto aut ex parte textus vetustissimos
continent.
Eorum melodiae de antiphonariis antiquis sumpsimus,
ut pro unaquaque antiphona et hymno proprio loco annotatur.
Quia melodiae tradendo variabantur et versio recentior nonnumquam
pulchrior et cantabilior invenitur, non semper versionem
vetustissimam reddimus.
(Quis illam maxime authenticam et restituendam putat,
in libris doctorum inveniet.)

Officia permulta aut nova aetate sunt indulta
(ut S. Ioannis Nepomuceni),
aut amplificata,
aut graviter reformata (ut S. Ludmilae, S. Venceslai, S. Procopii).
Ubi potuimus, ea in libris retentioribus, aut impressis
(pro S. Ioanne Nepomuceno et partim pro Ss. Cyrillo et Methodio),
aut manu scriptis invenimus et inventa reddimus.

\section*{De cantibus accommodandis}
A saeculo XVII unaquaeque fere editio \emph{Officiorum propriorum}
mutationes aliquas in officia vetera introduxit.
Permultae antiphonae vetustissimae abbreviatae sunt aut
alio modo immutatae.
In melodiis ad textus immutatos accommodandis semper studuimus
maximam partem melodiae veteris praeservare.
De omnibus immutationibus proprio loco rationem reddimus.

\section*{De cantibus componendis}
Officia nonnulla, praesertim saeculo XIX composita et indulta,
notis musicis numquam instructa sunt.
Pro his modulationes proprias composuimus, in stylo cantus gregoriani
et/aut cantuum veterorum Ecclesiae Pragensis.
Ubi melodia propria nondum composita est, textum saltem reddimus.

\section*{De libri huius partibus}
Liber hic quatuor habet partes.
Pars prima continet omnia officia hoc numero et ordine disposita
ut anno 1912, cum Antiphonale novum restitutum
cum dispositione nova psalterii in lucem editum est.
Pars secunda continet officia nova, amplificata aut mutata,
sed etiam praescribit, quomodo alia officia cum
Calendario Romano anno 1960 reformato adhibenda sunt.
Quia nonnullae antiphonae aut officia tota plurima cantus versione
gaudent, pars tertia continet versiones, quae loco eorum,
quae prima parte impressae sunt, ad libitum adhiberi possunt.
In quarta parte praesertim de antiphonis responsoriisque,
quos ipsi notis musicis adornavimus, rationem reddimus.


\part{Proprium Pragense ad~Antiphonale Sacrosanctae Romanae Ecclesiae
  a Pio Papa~X anno 1912 editum}

\section*{Kalendarium perpetuum Provinciae Pragenae}
\input{calendarium}
\cleardoublepage

Officium olim plenum Bohemiae cantabatur,
de quo post receptionem Breviarii Romani duae solum antiphonae in proprium provinciale
sunt receptae.
Textus ambarum hac occasione abbreviatus fuit.\edissue{Check the exact history of edits, I haven't excerpted it before}


\vfill
\clearpage

Hymnos laudum vesperarumque, qui in proprium pragense ab editione
1643 accepti sunt, David Drachovský de Hornstein, archidiaconus Plsnensis
(ob. post 1624) scripsit.
Omnes cantus alii (etiam hymnus matutini) anno 1865 de novo sunt scripti,
historiam rythmicam vetustissimam \emph{Adest dies laetitiae}
totum eradicantes.


\part{Accommodationes pro Calendario Romano anno 1960 reformato}

% TODO: calendarium

\begin{caputFesti}
  \diesFesti{Die 27 Septembris}
  \nomenFesti{Ss. Cosmae et Damiani}
  \descriptioFesti{Martyrum}
  \dignitasIII
\end{caputFesti}

\horaLaudes

\rubrica{Versiculus}
Exsultábunt sancti.
\rubrica{Ad Benedictus antiphona}
Beáti Mártyres cum duceréntur, \pageref{cosma:laudes}.
\rubrica{Oratio}
Magníficet te.

\begin{caputFesti}
  \diesFesti{Die 28 Septembris}
  \nomenFesti{S. Venceslai}
  \descriptioFesti{Ducis et Martyris, Patroni principalis Bohemiae}
  \dignitasI
\end{caputFesti}

\rubrica{Omnia ut supra, \pageref{venceslaus:vesperae1},
  tantum Commemoratio Octavae Dedicationis Ecclesiae
  Cathedralis Litomericensis non fit (festum Dedicationis
  non habet Octavam in Calendario reformato)
  et in I Vesperis fit Commemoratio Ss. Cosmae et Damiani.
  Antiphona}
Beáti Mártyres Christi.
\rubrica{Versiculus}
Laetámini in Dómino.
\rubrica{Oratio ut in festo, \pageref{cosma:vesperae1}.}


%\part{Aliae cantuum versiones ad libitum adhibendae}

\part{Annotationes}

\input{annotationes}

%\part{Indices}

\printbibliography

\end{document}

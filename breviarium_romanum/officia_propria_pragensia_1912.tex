\documentclass[12pt, a5paper, twoside]{book}

% To compile:
%
% $ xelatex filename
% $ biber filename
% $ xelatex filename

\usepackage{fontspec}
\setmainfont[Ligatures={TeX}]{Junicode}

\usepackage{csquotes}
\usepackage{polyglossia}
\usepackage[left=2cm, right=1.5cm, top=1.5cm, bottom=1.5cm, marginparsep=3mm]{geometry}

\usepackage[
  backend=biber,
  style=authortitle,
  sortlocale=la_LA,
  maxnames=3,
  firstinits=true,
]{biblatex}

\usepackage[show]{ed} % editorial annotations
\usepackage[hidelinks]{hyperref}
\usepackage{multicol}
\usepackage{parcolumns}
\usepackage{lettrine}

\usepackage{etoolbox}
\usepackage{luatextra}
\usepackage{graphicx} % support the \includegraphics command and options
\usepackage{gregoriotex} % for gregorio score inclusion

% polyglossia settings
\setmainlanguage{latin}

\newenvironment{caputFesti}{\begin{center}}{\end{center}}

% pieces of the title of a feast
\newcommand{\diesFesti}[1]{{\small #1} \vspace{2mm}\\}
\newcommand{\nomenFesti}[1]{\textbf{\Large #1}\vspace{3mm}\\}
\newcommand{\descriptioFesti}[1]{\textbf{#1}\\}
\newcommand{\dignitasFesti}[1]{\small #1}

% headings of various parts
\newcommand{\hora}[1]{\vspace{5mm}\noindent\textbf{#1}\vspace{2mm}}
\newcommand{\horaVesperaeI}{\hora{In I Vesperis}}
\newcommand{\horaLaudes}{\hora{In Laudibus}}
\newcommand{\horaLaudesEtHorae}{\hora{Ad Laudes et per Horas}}
\newcommand{\horaTertia}{\hora{Ad Tertiam}}
\newcommand{\horaSexta}{\hora{Ad Sextam}}
\newcommand{\horaNona}{\hora{Ad Nonam}}
\newcommand{\horaVesperaeII}{\hora{In II Vesperis}}

\newcommand{\parsHorae}[1]{\textbf{#1}\vspace{1mm}}
\newcommand{\parsOratio}{\begin{center}\emph{Oratio}\end{center}}
\newcommand{\parsCapitulum}[1]{\hspace{1.2cm}\textsc{Capitulum.}\hfill #1\hspace{1cm}}

\newcommand{\rubrica}[1]{{\small \emph{#1}}}

\newcommand{\utInBreviarioPraeter}{%
  \rubrica{Omnia ut in Antiphonario sub hac die, praeter sequentia.}}
\newcommand{\perDominum}{Per Dóminum.}

\newcommand{\versiculus}[2]{\noindent ℣. #1\\℟. #2}

% calendar
\newcommand{\calMonth}[1]{\begin{center}\textbf{#1}\end{center}}


\bibliography{bibliography}
\bibliography{sources}

\setcounter{secnumdepth}{3}
\setcounter{tocdepth}{3}

\grechangestyle{annotation}{\begin{footnotesize}}[\end{footnotesize}]
\grechangestyle{initial}{\fontsize{32}{32}\selectfont}

\title{Officia propria ecclesiasticae provinciae Pragensis}

\begin{document}

\pagestyle{empty}

\setlength{\parindent}{0.5cm}

\maketitle

\cleardoublepage

\pagestyle{plain}

\chapter*{Ratio huius editionis}

Breviarium proprium ecclesiae Pragensis ter
(1502, 1509, 1517)\footcite[242]{bohatta}
typis impressum est; post Breviarii Romani receptionem
saeculo XVII unumquidque saeculum plurimis editionibus
libelli \emph{Officia propria ecclesiasticae provinciae Pragensis} gaudebat.
Sed notae musicae usque ad dies nostros numquam typis sunt impressae
et tantum magnis cum difficultatibus in libris manu scriptis
inveniri possunt. Ideo nos antiphonalia vetustissima recentioresque
invenimus et comparavimus, et quos in illis invenimus,
cantus hic offere audemus.

\section*{De textibus et calendario}
Calendarium atque textus officiorum sumuntur de editionibus
\emph{Officiorum propriorum ecclesiasticae provinciae Pragensis}
anno 1912 et ultra editis.
Ubi post annum 1912 festum novum indultum, promotum, demotum
aut officium vetus mutatum est, hoc semper annotatum est,
cum anno quo mutatio haec occurrit.

Quia post annum 1960, quo Calendarium Breviariumque Romanum
graviter reformata sunt, nova \emph{Officia propria Pragensia}
pro adversitate temporis non sunt edita,
in secunda parte libri officia propria huic novo ordine Breviarii
accommodata proponimus. Sed nota quod propositio haec nullam habet
ecclesiasticam approbationem.

\section*{De fontibus musicis adhibitis}
Officia nonnulla, ex. g. S. Viti, S. Adalberti, Ss. Cosmae et Damiani,
etiam post omnes instaurationes mutationesque, quibus
Breviarii proprium Pragense a saeculo XVII usque ad dies nostros
subiectum est, ex toto aut ex parte textus vetustissimos
continent.
Eorum melodiae de antiphonariis antiquis sumpsimus,
ut pro unaquaque antiphona et hymno proprio loco annotatur.
Quia melodiae tradendo variabantur et versio recentior nonnumquam
pulchrior et cantabilior invenitur, non semper versionem
vetustissimam reddimus.
(Quis illam maxime authenticam et restituendam putat,
in libris doctorum inveniet.)

Officia permulta aut nova aetate sunt indulta
(ut S. Ioannis Nepomuceni),
aut amplificata,
aut graviter reformata (ut S. Ludmilae, S. Venceslai, S. Procopii).
Ubi potuimus, ea in libris retentioribus, aut impressis
(pro S. Ioanne Nepomuceno et partim pro Ss. Cyrillo et Methodio),
aut manu scriptis invenimus et inventa reddimus.

\section*{De cantibus accommodandis}
A saeculo XVII unaquaeque fere editio \emph{Officiorum propriorum}
mutationes aliquas in officia vetera introduxit.
Permultae antiphonae vetustissimae abbreviatae sunt aut
alio modo immutatae.
In melodiis ad textus immutatos accommodandis semper studuimus
maximam partem melodiae veteris praeservare.
De omnibus immutationibus proprio loco rationem reddimus.

\section*{De cantibus componendis}
Officia nonnulla, praesertim saeculo XIX composita et indulta,
notis musicis numquam instructa sunt.
Pro his modulationes proprias composuimus, in stylo cantus gregoriani
et/aut cantuum veterorum Ecclesiae Pragensis.
Ubi melodia propria nondum composita est, textum saltem reddimus.

\section*{De libri huius partibus}
Liber hic quatuor habet partes.
Pars prima continet omnia officia hoc numero et ordine disposita
ut anno 1912, cum Antiphonale novum restitutum
cum dispositione nova psalterii in lucem editum est.
Pars secunda continet officia nova, amplificata aut mutata,
sed etiam praescribit, quomodo alia officia cum
Calendario Romano anno 1960 reformato adhibenda sunt.
Quia nonnullae antiphonae aut officia tota plurima cantus versione
gaudent, pars tertia continet versiones, quae loco eorum,
quae prima parte impressae sunt, ad libitum adhiberi possunt.
In quarta parte praesertim de antiphonis responsoriisque,
quos ipsi notis musicis adornavimus, rationem reddimus.


\part{Proprium Pragense ad~Antiphonale Sacrosanctae Romanae Ecclesiae anno 1912 editum}

\section*{Kalendarium perpetuum Provinciae Pragenae\\1915+1928}

{\footnotesize
  Fontes adhibiti:
  \emph{Officia propria: pars hiemalis,} Kotrba Pragae 1915.
  \emph{Officia propria: pars autumnalis,} Kotrba Pragae 1915.
  \emph{Officia propria: pars verna,} Pustet Ratisbonae 1928.
  \emph{Officia propria: pars aestiva,} Pustet Ratisbonae 1928.
}

\input{calendarium/1915}
\cleardoublepage

\section*{Kalendarium proprium Provinciae Pragenae\\1949}

{\footnotesize
  Kalendarium sequens tantum festa propria provinciae Pragenae
  continet.

  Fontes adhibiti:
  \emph{Officia propria: pars hiemalis,} Gottmer Harlemi 1949.
  \emph{Officia propria: pars autumnalis,} Gottmer Harlemi 1949.
  \emph{Officia propria: pars verna,} Gottmer Harlemi 1949.
  (Parte aestivali, quam usque ad hunc diem locare non potuimus,
  kalendarium caret.)
}

\input{calendarium/1947}
\cleardoublepage

% incipitur a parte hiemali

\begin{caputFesti}
  \diesFesti{Die 1. Decembris}
  \dioecPraga
  \nomenFesti{B. Edmundi Campion}
  \descriptioFesti{Martyris e S. J.}
  \dignitasFesti{Semiduplex}
\end{caputFesti}

\rubrica{In II Vesperis praecedentis fit Com.
  per antiphonam}
Iste sanctus pro lege.
\rubrica{et versum}
Glória et honóre.

\parsOratio
Deus, qui verae fídei
et Sedis Apostólicae primátui propugnándo
beátum Mártyrem tuum Edmúndum
invícta fortitúdine roborásti:
ejus précibus exorátus,
nostrae, quaésumus, infirmitáti succúrre;
ut fortes in fide adversário resístere usque in finem valeámus.
Per Dóminum.

\rubrica{Et fit commemoratio feriae.}

\horaNocturnusII

\parsLectio{iv}

\lettrine{E}{dmúndus} Cámpion Londínii in Anglia natus,
humanióribus lítteris in Oxoniénsi academía,
tum sacris disciplínis Duáci in Anglórum seminário óptime excúltus,
Romae demum Societáti Jesu nomen dedit,
Pragae inter novítios recéptus,
primus Congregatiónis Beátae Maríae Vírginis praeses eléctus est,
et sacerdótio auctus per duos annos sacros sermónes habébat.
In pátriam Summi Pontíficis jussu,
una cum Robérto Persons, revérsus,
paucis ménsibus ea perfécit exémplo vitae,
excellénti doctrína atque agéndi dexteritáte,
ut ómnium ad se ánimos convérterit,
catholicórum quidem, ut eum audírent servaréntque,
inimicórum ut pérderent.

\rubRespDeCommuni{Honéstum fecit illum.}

\parsLectio{v}

\lettrine{I}{ndício} iniquíssimi proditóris detéctus,
mánibus post tergum revínctis Londínium addúcitur
abreptúsque in cárcerem conféstim in equúleum tóllitur,
atque ádeo crudéliter torquétur,
ut quassáto córpore paene semivívus jacéret.
Postrémo, licet nullo legítimo judício convíctus,
damnátur crudelíssimo supplícii génere,
quo perduelliónis rei plecti in Anglia solébant,
et impósitus crate vimínea ad Tybúrnum,
supplícii locum infámem, raptátur.

\parsLectio{vi}

\lettrine{E}{} crate in plaustrum subjéctum patíbulo sublátus
insertóque in láqueum collo,
circumfúsam úndique multitúdinem allocútus est,
seque cathólicum sacerdótem proféssus,
mortem pro fídei defensióne optatíssimam ultro se obíre affirmávit;
quandóquidem nullum sibi neque proditiónis neque conspiratiónis
crimen in regínam aut in pátriam óbjici posset.
Martýrium fecit Kaléndis Decémbris
anno millésimo quingentésimo octogésimo primo,
aetátis suae quadragésimo secúndo.
Cujus mártyris ejúsque sociórum cultum Gregórii décimi tértii
auctoritáte propósitum probatúmque,
Leo décimus tértius e senténtia sacrae Rituum Congregatiónis
sollémni décreto confirmávit quinto Idus Decémbris
anno millésimo octingentésimo octogésimo sexto.

\horaNocturnusIII

\parsEvangelium{Matthaéum}

\parsLectioEv{Cap. 10, 34-42}

\lettrine{I}{n} illo témpore:
Dixit Jesus discípulis suis:
Nolíte arbitrári, quia pacem vénerim míttere in terram:
non veni pacem míttere sed gládium.
Et réliqua.

\parsHomilia{sancti Hilárii Epíscopi}{Can. 10}

\lettrine{Q}{uae} ista divísio est?
inter prima enim legis praecépta accépimus:
Honóra patrem tuum et matrem tuam;
et ipse Dóminus ait:
Pacem meam do vobis, pacem meam relínquo vobis.
Quid sibi vult missus pótius gládius in terram,
et separátus a patre fílius, et fília a matre,
et nurus advérsus socrum, et hóminis doméstici ejus inimíci?
Igitur exínde pública auctóritas impietáti proferétur.
Ubíque ódia, ubíque bella et gládius Dómini inter patrem et fílium,
et inter fíliam matrémque desaéviens.

\rubRespDeCommuni{Coróna áurea.}

\parsLectio{viii}

\lettrine{G}{ládius} telórum ómnium telum acutíssimum est,
in quo sit jus potestátis, et judícii sevéritas,
et animadvérsio peccatórum.
Et hujus quidem teli nómine novi Evangélii praedicatiónem
appellátam frequens in Prophétis auctóritas est.
Dei igitur verbum nuncupátum meminérimus in gládio:
qui gládius missus in terram est, idest,
praedicátio ejus hóminum córdibus infúsa.
Fitque gravis in domo una dissénsio,
et doméstica novo hómini erunt inimíca:
quia ille per verbum Dei divísus ab illis,
manére intérior et extérior, idest,
et corpus et ánima, in spíritus novitáte gaudébit.

\parsLectio{ix}

\lettrine{P}{ergit} deínde eódem praeceptórum et intellegéntiae
decúrsu. Nam, posteáquam relinquénda ómnia,
quae in saéculo caríssima sunt, imperáverat, adjécit:
Qui non áccipit crucem suam, et séquitur me, non est me dignus:
quia Qui Christi sunt, crucifixérunt corpus cum vítiis
et concupiscéntia.
Et indígnus est Christo, qui non crucem suam, in qua compátimur,
commórimur, consepelímur, conresúrgimus, accípiens,
Dóminum sit secútus, in hoc sacraménto fídei spíritus novitáte
victúrus.

\parsTeDeum

\rubrica{Vesperae a capitulo de sequenti, commemoratio praecedentis:
  antiphona}
Qui vult veníre post me.
\rubrica{versus}
Justus ut palma.
\rubrica{Oratio ut supra. Deinde fit Com. feriae.}

\begin{caputFesti}
  \diesFesti{Die 6 Decembris}
  \dioecBud
  \nomenFesti{S. Nicolai}
  \descriptioFesti{Episcopi et Confessoris, Titularis Ecclesiae Cathedralis}
  \dignitasFesti{Duplex 1 classis cum Octava communi}
\end{caputFesti}

\rubrica{Omnia etiam Lectiones 1 Nocturni}
Fidélis sermo,
\rubrica{de Communi Confessoris Pontificis,
  praeter ea, quae in Breviario habentur propria.}

\rubrica{In Laudibus Commemoratio Feriae; in II Vesperis
  Commemoratio sequentis et Feriae.}

\begin{caputFesti}
  \diesFesti{Die 13 Decembris}
  \dioecBud
  \nomenFesti{In Octava S. Nicolai}
  \descriptioFesti{Episcopi et Confessoris}
  \dignitasFesti{Duplex majus}
\end{caputFesti}

\rubrica{In I Vesperis fit Commemoratio
  Octavae Immaculatae Conceptionis
  et S. Luciae Virginis et Martyris et Feriae.}

\horaNocturnusII

\parsSermo{sancti Gregórii Papae}{Part. 2 Pastoralis, cap. 1}

\parsLectio{iv}

\lettrine{T}{antum} debet actiónem pópuli áctio transcéndere praésulis,
quantum distáre solet a grege vita pastóris.
Opórtet namque, ut metíri se sollícite stúdeat,
quanta tenéndae rectitúdinis necessitáte constríngitur,
sub cujus aestimatióne pópulus grex vocátur.
Sit ergo necésse est cogitatióne mundus,
actióne praecípuus,
discrétus in siléntio,
útilis in verbo,
síngulis compassióne próximus,
prae cunctis contemplatióne suspénsus,
bene agéntibus per humilitátem sócius,
contra delinquéntium vítia per zelum justítiae eréctus,
internórum curam in exteriórum occupatióne non mínuens,
exteriórum providéntiam in internórum sollicitúdine non relínquens.

\rubRespDeCommuni{Invéni David.}

\parsLectioAnnot{v}{Part. 1 Pastor. Cap. 9 et 10}

\lettrine{C}{onsiderándum} quoque est, quia cum curam pópuli eléctus
praésul súscipit, quasi ad aegrum médicus accédit.
Si ergo adhuc in ejus córpore passiónes vivunt,
qua praesumptióne percússum medéri próperat,
qui in fácie vulnus portat?
Ille ígitur modis ómnibus debet ad exémplum bene vivéndi pértrahi,
qui cunctis carnis passiónibus móriens,
jam spiritáliter vivit,
qui próspera mundi postpónit,
qui nulla advérsa pertiméscit,
qui sola intérna desíderat;
cujus intentióni bene cóngruens,
nec omníno per imbecillitátem corpus,
nec valde per contumáciam repúgnat spíritus:
qui ad aliéna cupiénda non dúcitur,
sed própria largítur.

\parsLectioAnnot{v}{Ibidem Capite 8}

\lettrine{U}{nde} ipsum quoque Episcopátus offícium boni óperis
expressióne definítur, cum dícitur:
Si quis Episcopátum desíderat, bonum opus desíderat.
Ipse ergo sibi testis est, quia Episcopátum non áppetit,
qui non per hunc boni óperis ministérium,
sed honóris glóriam quaerit.
Sacrum quippe offícium non solum non díligit omníno,
sed nescit, qui ad culmen regíminis anhélans,
in occúlta meditatióne cogitatiónis,
ceterórum subjectióne páscitur,
laude própria laetátur,
ad honórem cor élevat,
rerum affluéntium abundántia exsúltat.
Mundi ergo lucrum quaéritur sub ejus honóris spécie,
quo mundi déstrui lucra debúerant.

\begin{caputFesti}
  \diesFesti{Die 22 Decembris}
  \dioecBud
  \nomenFesti{In Dedicatione Ecclesiae Cathedralis}
  \dignitasFesti{Duplex I classis cum Octava communi,
    de qua, ratione temporis, fit tantum die Octava}
\end{caputFesti}

\rubrica{In I Vesperis
  Commemoratio S. Thomae Apostoli
  et Feriae}
O Oriens.

\rubrica{Omnia ut in Communi Dedicationis Ecclesiae.
  In Laudibus Commemoratio Feriae.}

\rubrica{In II Vesperis Commemoratio Feriae}
O Rex.

\begin{caputFesti}
  \diesFesti{Die 29 Decembris}
  \dioecBud
  \nomenFesti{In Octava Dedicationis Ecclesiae Cathedralis}
  \dignitasFesti{Duplex majus}
\end{caputFesti}

\rubrica{In II Vesperis Ss. Innocentium Martyrum
  Commemoratio Octavae Dedicationis e I Vesperis
  ut in Communi Dedicationis;
  deinde Commemoratio S. Thomae Episcopi et Martyris
  et Octavae Nativitatis.}

\rubrica{Lectiones II et III Nocturni ut in Communi Dedicationis
  Ecclesiae die Octava.}

\rubrica{Pro S. Thoma Episcopo et Martyre:}

\parsLectio{ix}

Thomas, Londíni in Anglia natus,
ántea regni cancellárius, Theobáldo succéssit Cantuarénsi epíscopo.
In episcopáli offício fortis et invíctus,
leges utilitáti ac dignitáti ecclesiásticae repugnántes,
ab Henríco secúndo rege latas,
nullis fractus suis ac suórum incómmodis,
acceptáre rénuit.
Quare próxime conjiciéndus in cárcerem, clam recéssit;
et primo Pontiníaci apud mónachos Cisterciénses,
deínde apud Ludovícum regem Gálliae se cóntulit.
Ab exsílio revocátus, paulo post calúmniam apud regem ita impétitur,
ut saépius conquererétur rex,
se in suo regno cum uno sacerdóte pacem habére non posse.
Hinc nefárii hómines, sperántes se gratum regi factúros,
Thomam in Cantuariénsi templo vespertínis horis óperam dantem aggrediúntur.
Qui cléricis templi áditus praeclúdere conántibus óbstitit, dicens:
Non est Dei ecclésia custodiénda more castrórum;
et ego pro Ecclésia Dei libénter mortem subíbo.
Tum ad mílites ait: Vos Dei jussu cavéte,
ne cuípiam meórum noceátis.
Deínde, flexis génibus, ecclésiam et seípsum Deo comméndans,
cápite pléctitur,
quarto Kaléndas Januárii, anno Dómini millésimo centésimo
septuagésimo primo.

\parsTeDeum

\rubrica{Ad Laudes fit Commemoratio S. Thomae Episcopi et Martyris
  et Octavae Nativitatis.}

\rubrica{In II Vesperis Commemoratio sequentis diei infra Octavam
  Nativitatis
  et S. Thomae.}

\begin{caputFesti}
  \diesFesti{Die 2 Martii}
  \nomenFesti{B. Agnetis de Bohemia}
  \descriptioFesti{Virginis Ordinis Clarissarum}
  \dignitasFesti{Duplex}

  [Nota: Anno 1989 a Ioanne Paulo papa II est canonizata.]
\end{caputFesti}

\parsOratio

Deus, qui beátam Agnétem Vírginem
per regálium deliciárum contémptum
et húmilem tuae crucis sequélam
ad caelum sublimásti:
tríbue nobis, quaésumus,
ut ejus précibus et imitatióne
aetérnae glóriae mereámur esse partícipes.
Qui vivis.

\horaNocturnusII

\parsLectio{iv}

\lettrine{A}{gnes} Přemyslái Ottokári primi regis Bohemórum fília,
Pragae in pervigílio sanctae Vírginis et Mártyris Agnétis
in lucem édita fuit.
Bolesláo Silésiae ducis fílio deínde desponsáta a paréntibus,
et in monastério Trebnicénsi vírginum Cisterciénsium
prope Vratisláviam,
curánte beáta Hedwíge ducíssa collocáta,
ibi prima sanctitátis fundaménta jecit.
Cum triénnium in eo exegísset, defúncto sponso,
Dóxanam ad sanctimoniáles Praemonstraténses migrávit,
inter quas omni virtútum génere pro ténerae aetátis módulo enítuit.
Aliórum quoque núptias, non modo regum, sed et imperatóris,
non semel póstea oblátas constánter declinávit,
et ad Príncipem Austriae, áulicis erudiénda consuetudínibus missa,
non terréni sed caeléstis regni gustándis ánimum virésque omnes inténdit.
Indeféssa pérmanens in oratióne,
carnem suam jejúniis atque áliis asperitátibus edómuit;
erga páuperes vero et oppréssos téneram commiseratiónem aeque
ac profúsam liberalitátem exhíbuit.
Eándem vitae ratiónem ferventióri stúdio servávit in patérna domo,
cum quólibet sponsálium vínculo solúta esset;
donec, Summo Pontífice Gregório nono adjuvánte,
a régio fratre suo líbere sequéndi divínum sponsum Jesum Christum
et coenóbium ingrediéndi facultátem impetrávit.

\rubRespDeCommuni{Propter veritátem.}

\parsLectio{v}

\lettrine{A}{ttígerat} illis diébus Bohémiam fama novi reguláris
institúti a beáta Clara Assisiénsi cónditi:
qua permóta íllico Agnes mundánis omníno valedícere próperat.
Amplum paupéribus excoléndis in urbe Pragéna hospítium éxstruit,
recentíque Crucigenórum cum stella rúbea nuncupatórum órdini
regéndum commíttit.
Primum dein eádem in urbe Clarissárum monastérium érigit,
beátae ipsíus Agnétis nómine póstea insignítum;
illúdque tradit sanctimoniálibus virgínibus,
quas proptérea ab eádem institutríce Clara ad se mitti postuláverat.
Hisce mox et ipsa sociáta hábitu et sacro velámine ibi assúmpto,
álias nobilíssimas vírgines secum addúxit
ad novum severióris vitae institútum amplecténdum.
Ab ipso autem Gregório nono deínceps monastérii Abbatíssa constitúta
religiósam famíliam sanctíssime gubernávit.
Quamvis enim esset ómnium prima,
non áliter céteris praeésse visa est,
nisi praecláro devotiónis, oboediéntiae, castitátis,l
sui abnegatiónis ac demissiónis exémplo.

\parsLectio{vi}

\lettrine{S}{ed} vírginem humilitátis amantíssimam
honórum véluti postrémae relíquiae perturbábant.
Quare, brevi transácto témpore, Abbatíssae títulum recusávit,
ac tantúmmodo soror major monastérii deínceps vóluit appellári.
Oblátas a régio fratre divítias accípere rénuit,
nullúmque a soróribus retinéri permísit temporálium bonórum domínium.
Supernárum visiónum dono et grátia curatiónum ditáta a Deo
fuísse perhibétur.
Jamque illi soli vivens, cum in ómnium virtútum cultúra constánter
perseverásset,
tandem, cumuláta méritis, in caelum migrávit circa reparátae salútis
annum ducentésimum octogésimum supra millésimum,
aetáte fere octogenária.
Cultum ab immemorábili témpore beátae Agnéti praéstitum
Pius nonus Póntifex Máximus, ex Sacrórum Rítuum Congregatiónis consúlto,
apostólica auctoritáte ratum hábuit et confirmávit.

\rubrica{In III Nocturno Homilia in Evangelium}
Símile erit regnum caelórum,
\rubrica{de Communi Virginum 1 loco.}

\begin{caputFesti}
  \diesFesti{Die 4 Martii}
  \nomenFesti{In Translatione S. Venceslai}
  \descriptioFesti{Regis, Martyris}
  \dignitasFesti{Duplex majus}
\end{caputFesti}

\parsOratio

Concéde, quáesumus,
plebi tuae, omnípotens Deus:
ut quae beáti Venceslái Mártyris tui translatióne laetátur in terris,
ejúsdem aetérna societáte perfruátur in caelis.
Per Dóminum.

\rubrica{Et fit Commemoratio S. Casimiri Confessoris
  per Antiphonam}
Similábo eum.
\rubrica{et Versum}
Amávit eum Dóminus.

\parsOratio

Deus, qui inter regáles delícias et mundi illécebras,
sanctum Casimírum virtúte constántiae roborásti:
quaésumus;
ut, ejus intercessióne, fidéles tui terréna despíciant,
et ad caeléstia semper aspírent.

\rubrica{Deinde fit Commemoratio S. Lucii Papae et Martyris
  per Antiphonam}
Qui odit.
\rubrica{et Versum}
Justus ut palma florébit.

\parsOratio

Deus, qui nos beáti Lúcii Mártyris tui atque Pontíficis
ánnua sollemnitáte laetíficas:
concéde propítius;
ut, cujus natalícia cólimus,
de ejúsdem étiam protectióne gaudeámus.
Per Dóminum.

\horaNocturnusII

\parsLectio{iv}

\lettrine{B}{eáti} Ducis et Mártyris Venceslái corpus,
cum triénnio jam Bolesláviae in aede sanctórum Cosmae et Damiáni jacéret,
atque ibídem miráculis coruscáret,
Pragénsem in urbem Boleslái fratris jussu translátum est.
Qui vivénti manus intúlerat, in mórtuum étiam ímpius fuit.
Itaque plaustro impósitum clam ómnibus,
vitáta pópuli frequéntia, uníus noctis spátio
inglórium et inhonorátum deférri a suis imperávit,
poena aurígae impósita,
ut, nisi ante lucem Pragam pervenísset, cápite plecterétur.
Sed haec consília Deus admirabíliter dissipávit.

\rubRespDeCommuni{Honéstum fecit illum.}

\parsLectio{v}

\lettrine{C}{onténdunt,} uti imperátum fúerat, festinatióne summa.
Binos flúvios, aquis tunc praeter sólitum redundántes,
equórum pédibus tantum madefáctis, transmíttunt.
Ubi ad cárcerem, arci Pragénsi subjéctum, devéntum est:
ita equi constitére, ut inde dimovéri non possent,
nisi clara jam luce, cum plúrimi ingénti concúrsu conveníssent.
Quod Bolesláus máxime vitábat, ejúsque satéllites obstináte
pernegábant, planum est ómnibus factum, sacratíssimum beáti
Venceslái corpus ab ipsis deférri.

\parsLectio{vi}

\lettrine{A}{ccéssit} miráculum áliud, quo res illústrior fíeret.
Vincti ad unum omnes, qui cárcere custodiebántur,
divínitus rédditi libertáti: in cujus rei memóriam aedes,
sancti Venceslái nómine insígnis, ibídem constrúcta est.
Attónitis ob haec prodígia ómnibus, beáti Príncipis corpus
in sancti Viti templum, quod ipse olim fundáverat,
fuit illátum. Rejéctis invólucris, una dempta aurícula,
quae deínde cápiti admóta moménto coáluit,
artus plane íntegros obdúctis vúlnerum cicatrícibus
summa cum admiratióne repériunt.
Itaque inter hymnos et cántica, cum ómnia fácibus collucérent,
sacerdótum clericorúmque ópera, ea, qua par erat,
veneratióne recónditum fuit.

\rubrica{In III Nocturno de Homilia in Evangelium}
Si quis vult post me venire,
\rubrica{de Communi unius Martyris 2 loco.}

\rubrica{Pro S. Casimiro Confessore:}

\parsLectio{ix}

\lettrine{C}{asimírus,} Polóniae regis fílius, a puerítia pietáte et bonis ártibus
instrúctus, juveníles artus áspero domábat cilício,
et assíduis extenuábat jejúniis.
In Christi contemplánda passióne assíduus,
oratiónis spíritum non relaxábat,
Cathólicam fidem promovére,
et Ruthenórum schisma abolére summópere stúduit.
Erga páuperes et calamitátibus oppréssos benéficus et miséricors,
patris et defensóris egenórum nomen obtínuit.
Virginitátem usque ad extrémum vitae términum constánter servávit
illaésam.
Consummátus in brevi, virtútibus et méritis plenus,
praenuntiáto mortis die, spíritum Deo réddidit,
anno aetátis vigésimo quinto.
Eum, miráculis clarum, Leo décimus in Sanctórum númerum réttulit.

\parsTeDeum

\rubrica{In Laudibus fit Commemoratio S. Casimiri Confessoris
  per Antiphonam}
Euge, serve bone.
\rubrica{et Versum}
Justum dedúxit.
\rubrica{Oratio ut supra.}

\rubrica{Deinde Commemoratio S. Lucii I Papae et Martyris
  pet Antiphonam}
Iste Sanctus.
\rubrica{et Versum}
Glória et honóre.
\rubrica{Oratio ut supra.}

\rubrica{In II Vesperis fit Commemoratio S. Casimiri Confessoris
  per Antiphonam}
Hic vir despíciens mundum.
\rubrica{et Versum}
Justum dedúxit.
\rubrica{Oratio ut supra.}

\begin{caputFesti}
  \diesFesti{Die 17 Martii}
  \nomenFesti{B. Joannis Sarcander}
  \descriptioFesti{Martyris}
  \dignitasFesti{Duplex}
\end{caputFesti}

\parsOratio
Deus, qui beátum Joánnem Mártyrem tuum
in confessióne verae fídei
et sacramentális siléntii custódia
virtúte constántiae roborásti:
praesta, quaésumus;
ut contra advérsa ómnia ejus muniámur exémplis,
et protegámur auxíliis.
Per Dóminum.

\rubrica{Et fit Commemoratio S. Patricii Episcopi et Confessoris
  per Antiphonam}
Sacérdos et Póntifex.
\rubrica{et Versum}
Amávit eum.

\parsOratio
Deus, qui ad praedicándam géntibus glóriam tuam
beátum Patrícium Confessórem atque Pontíficem
míttere dignátus es:
ejus méritis et intercessióne concéde;
ut, quae nobis agénda praécipis,
te miseránte, adimplére possímus.
(Per Dóminum.)

\rubrica{Deinde fit Commemoratio Feriae.}

\rubrica{In I Nocturno Lectiones}
Fratres: Debitóres,
\rubrica{de Communi plurimorum Martyrum cum Responsoriis unius Martyris.}

\horaNocturnusII

\parsLectio{iv}

\lettrine{J}{oánnes} Sarcánder,
Skočóviae in ducátu Teschinénsi superióris Silésiae nóbili génere ortus,
adhuc puer, amísso patre, tutélae matérnae créditus,
infensíssima cathólico nómini tempestáte,
piam in lítteris et móribus institutiónem quaéritans,
in Moráviam finitimásque regiónes conténdit.
Freibérgae primum consístit, deínde Olomúcium,
mox Pragam in Bohémia, postrémo Graécium in Stýria
ad humanióres et sacras disciplínas ediscéndas proficíscitur.
Sacerdótio itémque doctóris título auctus,
in Moráviam dénuo revérsus,
a Cardináli Epíscopo Olomucénsi cúriae Holešoviénsi post grassátam
ibi octogínta annos haéresim praefícitur.
Pastorále munus rite fidelitérque gessit;
verbo et exémplo pietáti fovéndae et amplificándae,
aberrántibus ad cathólicam unitátem revocándis ac recipiéndis,
ecclesiásticae immunitátis júribus tuéndis,
orthodóxis doctrínis de sacraméntis Paeniténtiae et Eucharístiae,
a sacrosáncto Concílio Tridentíno tráditis ac confirmátis,
ácriter propugnándis, óperam sédulo navávit;
ex quo ómnium haereticórum invídias et simultátes in se convértit,
Bohémica seditióne gliscénte, tantísper Holešóvio migrávit,
et in Polóniam ad beátam Maríam Vírginem Czenstochóvii
ex voto secéssit:
sed morae impátiens, ac desidério fidélium in fide confirmandórum
incénsus, cum circa Silesiános fines diu oberrásset,
íterum Holešóvium se recépit.
Proscríptus brevi, et in jus vocátus,
in arce Tovačóvii se subrípuit,
donec in silva Olomúcio próxima próditus,
et a satellítibus abréptus,
contumélias et calúmnias haereticórum mira ánimi constántia
ac mansuetúdine perpéssus, in víncula conjéctus est.

\rubRespDeCommuni{Honéstum fecit illum.}

\parsLectio{v}

\lettrine{I}{nstitúta} inquisitióne, quater in judícium addúctus,
coram saevíssimis optimátibus Moravórum convíciis,
maledíctis et exsecratiónibus excéptus:
ter, suspénsus in equúleo, arctíssime, ad sex continéntes horas,
attráctus extentúsque est:
bis vero admótis ad látera per quinque horas fácibus sebo,
súlphure, resína imbútis:
tandem adhíbitis plumis, picis ac ólei liquámine íllitis et accénsis,
inque látera, ventrem, collum, axillásque impáctis,
ita adústus est, ut, depástis flamma cárnibus,
vix víscera inter costárum repágula cohiberéntur.
Hisce cruciátibus in Mártyrem animadvértunt haerétici
tum propter ódium disciplínae ac fídei cathólicae,
eas in regiónes invéctae, ac strénue propugnátae,
tum ut dynástae de Lóbkovic arcána,
in exomologési pándita, de cónscii sacerdótis péctore elícerent
atque extorquérent.
Frustra diu invícti Mártyris contra honórem sacraménti tentáta
constántia est:
nam patíbulum patiéntis factum est cáthedra docéntis,
ex qua protestatiónibus, adhortatiónibus et eníxis ad Deum précibus,
necnon jugi et validíssima sanctórum nóminum Jesu, Maríae et Annae
invocatióne, júdicum rábiem devícit et feritátem carníficum prostrávit.

\parsLectio{vi}

\lettrine{L}{ictóri} haerético et cárceri perpétuo tráditus,
in oratióne ac caeléstium rerum contemplatióne relíquum passiónis
suae tempus,
quod inter acerbíssimos dolóres ad trigínta tres dies perdúctum est,
insúmpsit.
Cotídie Horas Canónicas recitávit:
cumque ob diffráctos nervos et disrúptos totíus córporis artus
impos ad múnia vitae obeúnda redderétur,
linctu linguae opem mánuum in páginis verténdis supplébat.
Tandem die décima séptima Mártii
anni millésimi sexcentésimi vigésimi, annos natus quadragínta tres,
victrícem ánimam caelo inseréndam réddidit in illis verbis:
Convértere, ánima mea, in réquiem tuam,
quia Dóminus benefécit tibi:
quia erípuit ánimam meam de morte, óculos meos a lácrimis,
pedes meos a lapsu.
Ejus corpus ob inédiam et diutúrnam tolerántiam,
viscerúmque putrefactiónem foetens ac squálidum,
post mortem quodámmodo juvénta refloruísse,
et suávem odórem exhalásse compértum est,
ita ut illud Cathólici ad septem dies inhumátum serváverint.
Quod demum in Ecclésia Deíparae Vírginis sidéribus recéptae
Olomúcii cónditum est.
Antrum supplícii in Ecclésiam convérsum est;
ibi equúleus servátur septo conclúsus,
ne importúna fidélium pietáte in frusta comminuátur;
ibíque fons aquae perénnis vísitur,
qui ad levándam Mártyris sitim súbito scátuit,
valde céleber, quod ejus haustu febres fugántur.
Quem miráculis clarum Pius nonus Póntifex Máximus
anno millésimo octingentésimo sexagésimo
beatórum Caélitum albo accénsuit,
et inter cathólicae Ecclésiae mártyres réttulit.

\rubrica{In III Nocturno Homilia in Evangelium}
Nihil est opértum,
\rubrica{de Communi unius Martyris, 4 loco.}

\rubrica{IX Lectio de Homilia Feriae.}

\rubrica{In Laudibus fit Commemoratio S. Patricii Episcopi et
  Confessoris per Antiphonam}
Euge, serve bone.
\rubrica{et Versum}
Justum dedúxit.
\rubrica{Oratio ut supra.}

\rubrica{Deinde fit Commemoratio Feriae.}

\rubrica{Vesperae a Capitulo de sequenti,
  Commemoratio praecedentis per Antiphonam}
Qui vult veníre.
\rubrica{et Versum}
Justus ut palma.
\rubrica{Oratio ut supra.}

\rubrica{Deínde Commemoratio S. Patricii Episcopi et Confessoris
  per Antiphonam}
Amávit eum.
\rubrica{et Versum}
Justum dedúxit.
\rubrica{Oratio ut supra.}

Antiphona \emph{Gloria Christo Domino} vetustissima est,
antiphona \emph{Inclyte Martyr} vero recens est versio antiphonae veteris,
valde abbreviata atque immutata.
Omnes cantus alii anno 1865 proprio Pragensi additi sunt.

Hymnus \emph{Praesulum salve} magna ex parte desumptus est ex hymnis Georgii Bartholdi a Braitenberg,
canonici Pragensis.\footcite[134]{pontanus1590}

\begin{caputFesti}
  \diesFesti{Die 2 Maji}
  \nomenFesti{S. Sigismundi}
  \descriptioFesti{Regis, Martyris et Patroni Secundarii Regni}
  \dignitasFesti{Duplex majus}
\end{caputFesti}

% in the old Officia propria antiphon and verse from the Common
% is printed.

\parsOratio
Deus, qui hunc diem beáti Sigismúndi passióne consecrásti:
praesta, quaésumus;
ut, cujus sollémnia celebrámus in terris,
ejus apud te suffrágiis adjuvémur in caelis.
(Per Dóminum.)

\rubrica{Deinde Commemoratio S. Athanasii Episcopi, Confessoris
  et Ecclesiae Doctoris
  per Antiphonam}
O Doctor óptime.
\rubrica{et Versum}
Amávit eum.

\parsOratio
Exáudi, quaésumus, Dómine, preces nostras,
quas in beáti Athanásii, Confessóris tui atque Pontíficis,
sollemnitáte deférimus:
et, qui tibi digne méruit famulári,
ejus intercedéntibus méritis,
ab ómnibus nos absólve peccátis.
Per Dóminum.

\horaNocturnusII

\parsLectio{iv}

\lettrine{S}{igismúndus,} Gunebáldi Burgundiórum Regis fílius,
a téneris annis in cathólica religióne educátus,
ita Christiánae fídei fuit addíctus,
ut adoléscens factus, diu noctúque vigíliis, jejúniis et oratiónibus vacans,
non obscúra déderit sanctitátis indícia.
Mórtuo Gunebáldo, patérnum sceptrum adéptus,
zelo propagándae fídei succénsus,
summa cura summóque labóre, praesértim vero exémplo et virtútibus
regnum a ténebris infidelitátis ad lucem veritátis addúcere stúduit.
Religiónis quoque, ac devotiónis erga divínum cultum
et Sanctórum Ecclésias promovéndae causa,
monastérium Agaunénse, in quo Psalléntium Ordinem instítuit,
regáli munificéntia cum dómibus Basilicísque aedificávit,
et magnis redítibus locupletávit.

\rubRespDeCommuni{Lux perpétua.}

\parsLectio{v}

\lettrine{A}{míssa} prióri cónjuge fília Theodoríci Regis Itáliae,
ex qua fílium suscéperat nómine Sigerícum,
áliam duxit uxórem.
Hujus nequíssimis decéptus suasiónibus, fílium intérfici jussit.
Quo facto, Sigismúndus corde compúnctus, super cadáver próruens,
flere coepit amaríssime: deínde velut alter David,
severiórem poeniténtiae viam ingréssus, multis Sanctórum locis perlustrátis,
demum labóribus et inédia fessus, ad sepúlchra Sanctórum Agaunénsium pervéniens,
ibi per multos dies in fletu et jejúnio persevérans,
Deum eníxe rogábat, ut si quid adhuc pro consequéndo coeléstis pátriae
regno sibi superésset, misericórditer osténdere dignarétur.
Summa Dei bónitas, quae labóribus servi sui remuneratiónem
diútius non patiebátur différre,
ad palmam martýrii ipsum eátenus vocávit,
ut sanctórum Thebaeórum Mártyrum collégio,
quorum se offício in Dei láudibus sociáverat devotióne,
paradísi quoque sociarétur glória.

\parsLectio{vi}

\lettrine{N}{am} cum Franci Galliárum gentes et urbes depopularéntur,
Burgundiónibus, qui adhuc in infidelitáte persistébant,
sibi sociátis, Sigismúndum, qui ut barbarórum ferocitátem eváderet,
Vesállis montem petíerat, ibíque tonso crine,
et Religiónis hábitu suscépto, singuláriter habitábat,
a suis decéptum, et ad sepúlchra Sanctórum ductum,
una cum uxóre et fíliis capitáli senténtia adjudicátum,
in púteum véterem apud Colóniam vicum projecérunt.
Corpus ejus divína póstea revelatióne patefáctum,
indéque sublátum, et in Ecclésia Agaunénsi honorífice sepúltum,
miráculis claréscere coepit, ita Deo ejus sanctitátem comprobánte.
Póstea a Carólo Quarto Cáesare
ad Ecclésiam Metropolitánam Pragénsem translátum,
gloriósum ibídem túmulum invénit.

\horaNocturnusIII

\parsEvangelium{Joánnem}

\parsLectioEv{Cap. 15, 1-7}

\inIlloTempore{}
Dixit Jesus discípulis suis:
Ego sum vitis vera, et Pater meus agrícola est.
Et réliqua.

\parsHomilia{sancti Augustíni Epíscopi}{Tract. 81 in Joannem}

\lettrine{V}{item} se dixit esse Jesus,
et discípulos suos pálmites,
et agrícolam Patrem:
unde jam pridem, sicut potúimus, disputátum est.
In hac autem lectióne cum adhuc de se ipso, qui est vitis,
et de suis palmítibus, hoc est, discípulis, loquerétur:
Manéte, inquit, in me, et ego in vobis.
Non eo modo illi in ipso, sicut ipse in illis.
Utrúmque autem prodest non ipsi, sed illis:
ita quippe in vite sunt pálmites,
ut viti non cónferant, sed inde accípiant, unde vivant;
ita vero vitis in palmítibus,
ut vitále aliméntum subminístret eis, non sumat ab eis.

\rubRespDeCommuni{Ego sum vitis vera.}

\rubrica{Feria III et VI infra hebdomadam I et II post Octavam Paschae,
  quoties in I Nocturno Lectiones fuerint de Scriptura occurrenti
  cum suis Responsoriis de Tempore,
  loco VII Responsorii dicitur}
\rr{} Tristítia vestra, allelúja.
\rubrica{ut in III Nocturno Dominicae III post Pascham.}

\parsLectio{viii}

\lettrine{A}{c} per hoc, et manéntem in se habére Christum,
et manére in Christo, discípulis prodest utrúmque, non Christo.
Nam praecíso pálmite, potest de viva radíce álius polluláre;
qui autem praecísus est, sine radíce non potest vívere.
Dénique adjúngit, et dicit:
Sicut palmes non potest ferre fructum a semetípso,
nisi mánserit in vite;
sic nec vos, nisi in me manséritis.
Magna grátiae commendátio, fratres mei:
corda ínstruit humílium, ora óbstruit superbórum.

\rubrica{Pro S. Athanasio Episcopo, Confessore et Ecclesiae Doctore:}

\parsLectio{ix}

\lettrine{A}{thanásius,} epíscopus Alexandrínus,
cathólicae religiónis propugnátor acérrimus,
cum, adhuc diáconus, in Concílio Nicaéno Aríi impietátem repressísset,
tantum ódium Arianórum suscépit,
ut ex eo témpore et insídias molíri numquam destíterint.
In exsílium actus, in Gállia apud Tréviros exulávit.
Incredíbiles dein calamitátes perpéssus,
magnam orbis partem peragrávit;
ac saepe e sua ecclésia ejéctus, saepe étiam in eándem,
Júlii Románi Pontíficis auctoritáte atque decrétis concílii
Sardicénsis ac Jerosolymitáni restitútus est,
Ariánis intérea illi semper inféstis.
Dénique ex tot tantísque perículis divínitus eréptus,
Alexandríae mórtuus est sub Valénte.
Ejus vita et mors magnis nobilitáta est miráculis.
Multa pie et ad illustrándam cathólicam fidem praecláre scripsit,
sexque et quadragínta annos in summa témporum varietáte
Alexandrínam ecclésiam sanctíssime gubernávit.

\parsTeDeum

\rubrica{In Laudibus fit Commemoratio S. Athanasii
  per Antiphonam}
Euge, serve bone.
\rubrica{et Versum}
Justum dedúxit.
\rubrica{Oratio ut supra.}

\rubrica{Vesperae de sequenti, Commemoratio S. Sigismundi Martyris
  per Antiphonam}
Sancti et justi.
\rubrica{et Versum}
Pretiósa in conspéctu.
\rubrica{Oratio ut supra.}

\rubrica{Deinde fit Commemoratio S. Athanasii ut in Breviario.}

Officium anno 1741 a Benedicto papa XIV indultum fuit et
numquam\edissue{make sure that the lessons were not actually reformed}
immutatum.
Per totam fere Germaniam aliundeque diffusum, permultas habet
melodiarum versiones, praesertim pro vesperis.


\lacuna{magna}

Officium olim plenum Bohemiae cantabatur,
de quo post receptionem Breviarii Romani duae solum antiphonae in proprium provinciale
sunt receptae.
Textus ambarum hac occasione abbreviatus fuit.\edissue{Check the exact history of edits, I haven't excerpted it before}

Hymnos laudum vesperarumque, qui in proprium pragense ab editione
1643 accepti sunt, David Drachovský de Hornstein, archidiaconus Plsnensis
(ob. post 1624) scripsit.
Omnes cantus alii (etiam hymnus matutini) anno 1865 de novo sunt scripti,
historiam rythmicam vetustissimam \emph{Adest dies laetitiae}
totum eradicantes.


\lacuna{non parva}

\part{Accommodationes pro Calendario Romano anno 1960 reformato}

\section*{Kalendarium proprium Provinciae Pragenae\\1960}

{\footnotesize
  Kalendarium sequens tantum festa propria provinciae Pragenae
  continet.

  Fontes adhibiti:
  \emph{Directorium 1962,} Česká katolická Charita v Ústředním církevním nakladatelství v Praze.
  \emph{Directorium 1964,} Česká katolická Charita v Ústředním církevním nakladatelství v Praze.
}

\begin{caputFesti}
  \diesFesti{Die 27 Septembris}
  \nomenFesti{Ss. Cosmae et Damiani}
  \descriptioFesti{Martyrum}
  \dignitasIII
\end{caputFesti}

\horaLaudes

\rubrica{Versiculus}
Exsultábunt sancti.
\rubrica{Ad Benedictus antiphona}
Beáti Mártyres cum duceréntur, \pageref{cosma:laudes}.
\rubrica{Oratio}
Magníficet te.

\begin{caputFesti}
  \diesFesti{Die 28 Septembris}
  \nomenFesti{S. Venceslai}
  \descriptioFesti{Ducis et Martyris, Patroni principalis Bohemiae}
  \dignitasI
\end{caputFesti}

\rubrica{Omnia ut supra, \pageref{venceslaus:vesperae1},
  tantum Commemoratio Octavae Dedicationis Ecclesiae
  Cathedralis Litomericensis non fit (festum Dedicationis
  non habet Octavam in Calendario reformato)
  et in I Vesperis fit Commemoratio Ss. Cosmae et Damiani.
  Antiphona}
Beáti Mártyres Christi.
\rubrica{Versiculus}
Laetámini in Dómino.
\rubrica{Oratio ut in festo, \pageref{cosma:vesperae1}.}

\cleardoublepage

\begin{caputFesti}
  \diesFesti{Die 27 Septembris}
  \nomenFesti{Ss. Cosmae et Damiani}
  \descriptioFesti{Martyrum}
  \dignitasIII
\end{caputFesti}

\horaLaudes

\rubrica{Versiculus}
Exsultábunt sancti.
\rubrica{Ad Benedictus antiphona}
Beáti Mártyres cum duceréntur, \pageref{cosma:laudes}.
\rubrica{Oratio}
Magníficet te.

\begin{caputFesti}
  \diesFesti{Die 28 Septembris}
  \nomenFesti{S. Venceslai}
  \descriptioFesti{Ducis et Martyris, Patroni principalis Bohemiae}
  \dignitasI
\end{caputFesti}

\rubrica{Omnia ut supra, \pageref{venceslaus:vesperae1},
  tantum Commemoratio Octavae Dedicationis Ecclesiae
  Cathedralis Litomericensis non fit (festum Dedicationis
  non habet Octavam in Calendario reformato)
  et in I Vesperis fit Commemoratio Ss. Cosmae et Damiani.
  Antiphona}
Beáti Mártyres Christi.
\rubrica{Versiculus}
Laetámini in Dómino.
\rubrica{Oratio ut in festo, \pageref{cosma:vesperae1}.}


%\part{Aliae cantuum versiones ad libitum adhibendae}

\part{Annotationes}
\label{pars:annotationes}

\input{annotationes}

%\part{Indices}

\input{sourcesused}
\printbibliography[title=Fontes cantuum manuscripti, keyword=source, keyword=manuscript, heading=subbibliography]
\printbibliography[title=Fontes cantuum typis impressi, keyword=source, notkeyword=manuscript, heading=subbibliography]

\printbibliography[title=Alii libri adhibiti, notkeyword=source, heading=subbibliography]

\end{document}

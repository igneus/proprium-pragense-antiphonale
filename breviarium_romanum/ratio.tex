\chapter*{Ratio huius editionis}

Breviarium proprium ecclesiae Pragensis
ante Breviarii Romani receptionem ter
(1502, 1509, 1517)\footcite[242]{bohatta}
typis impressum est; post Breviarii Romani receptionem
saeculo XVII vero unumquidque saeculum plurimis editionibus
libelli \emph{Officia propria ecclesiasticae provinciae Pragensis} gaudebat.
Notae musicae autem, ad cantum omnium antiphonarum, hymnorum
responsoriorumque necessariae,
usque ad dies nostros numquam typis sunt impressae
et tantum magnis cum difficultatibus de libris variis,
praecipue de veteris codicibus manu scriptis,
colligi possunt. Ideo nos antiphonalia vetustissima recentioresque
invenimus et comparavimus, et quos in illis invenimus,
cantus hic offere audemus.

\section*{De textibus et calendario}
Calendarium atque textus officiorum sumuntur de editionibus
\emph{Officiorum propriorum ecclesiasticae provinciae Pragensis}
anno 1912 et ultra editis.
Ubi post annum 1912 festum novum indultum, promotum, demotum
aut officium vetus mutatum est, hoc semper annotatur,
cum anno quo mutatio haec occurrit.

Quia post annum 1960, quo Calendarium Breviariumque Romanum
graviter reformatum est, nova \emph{Officia propria Pragensia}
pro adversitate temporis non sunt edita,
in secunda parte libri officia propria huic novo ordine Breviarii
accommodata ponimus, secundum directoria post annum 1960 typis
impressa.

\section*{De libris quos cantor habere debet}
Liber hic modo appendix esse vult ad \emph{Antiphonale}
anno 1912 iussu Pii Papae X editum (quoad horas diurnas) et
ad \emph{Nocturnale Romanum} a Holger Peter Sandhofe exaratum
et anno 2002 editum (quoad matutinum).
Quicumque integrum officium divinum in festis ecclesiasticae provinciae
Pragenae propriis tam diurnum quam nocturnum secundum
libellum nostrum dicere vult, hos duos libros habere debet.
Cantus in his libris contenti hic de novo non imprimuntur,
sed tantum necessitas eos illic quaerere indicatur.
Cantus in aliis libris typis impressi, sed non in supranominato
Antiphonali et Nocturnali (ut officium
\enquote{B.~M.~V. omnium gratiarum mediatricis}, plurimis dioecesibus,
provinciis ac societatibus religiosis indultum), hic imprimuntur.

\section*{De fontibus musicis adhibitis}
Officia nonnulla, ex. g. S. Viti, S. Adalberti, Ss. Cosmae et Damiani,
etiam post omnes instaurationes mutationesque, quibus
Breviarii proprium Pragense a saeculo XVII usque ad dies nostros
subiectum est, ex toto aut ex parte textus vetustissimos
continent.
Eorum melodiae de antiphonariis antiquis sumpsimus,
ut pro unaquaque antiphona et hymno proprio loco annotatur.
Quia melodiae tradendo variabantur et versio recentior nonnumquam
pulchrior et cantabilior videbatur, non semper melodiam
vetustissimam reddimus.
(Quis illam maxime authenticam et restituendam putat,
in libris doctorum inveniet.)

Officia permulta aut nova aetate sunt indulta
(ut S. Ioannis Nepomuceni),
aut amplificata,
aut graviter reformata (ut S. Ludmilae, S. Venceslai, S. Procopii).
Ubi potuimus, ea in libris recentioribus, aut impressis
(pro S. Ioanne Nepomuceno et partim pro Ss. Cyrillo et Methodio),
aut manu scriptis invenimus et inventa reddimus.

\section*{De cantibus accommodandis}
A saeculo XVII unaquaeque fere editio \emph{Officiorum propriorum}
mutationes aliquas in textus veterorum officiorum introduxit,
textibus decantandis non parcens.
Permultae antiphonae vetustissimae abbreviatae sunt aut
alio modo immutatae.
In melodiis ad textus immutatos accommodandis semper studuimus
maximam partem melodiae veteris praeservare.
De omnibus immutationibus proprio loco rationem reddimus.

\section*{De cantibus componendis}
Officia nonnulla, praesertim saeculo XIX composita et indulta,
notis musicis numquam instructa sunt.
Pro his modulationes proprias composuimus, in stylo cantus gregoriani
et/aut cantuum veterorum Ecclesiae Pragensis.
Ubi melodia propria nondum composita est, textum saltem reddimus.

\section*{De libri huius partibus}
Liber hic
tres
%% quatuor
habet partes.
Pars prima continet omnia officia hoc numero et ordine disposita
ut anno 1912, cum Antiphonale novum restitutum
cum dispositione nova psalterii in lucem editum est.
Pars secunda ordinat, quomodo textus cantusque in prima parte
editi secundum Calendarium et Rubricas novas Breviarii Romani
anno 1960 promulgatas adhibendi sunt.
%% Quia nonnullae antiphonae aut officia tota plurima cantus versione
%% gaudent, pars tertia continet versiones, quae loco eorum,
%% quae prima parte impressae sunt, ad libitum adhiberi possunt.\edissue{Pars tertia est tantum in votis.}
In tertia parte de antiphonis responsoriisque,
quos aut textibus immutatis accommodavimus,
aut ipsi notis musicis adornavimus, rationem reddimus.
